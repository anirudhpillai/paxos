\newgeometry{top=2cm, left=1cm, right=1cm, bottom=2cm}
\begin{landscape}

\chapter{Disel Proof Code Listing}

The complete source code can be found at
\href{https://github.com/anirudhpillai/disel}{https://github.com/anirudhpillai/disel}.
The encoding for Paxos is in \texttt{disel/Examples/Paxos}. Here we are only
showing the code for the wrapper around the client application and the main
shim file for it.

\section{scripts/paxos.py}
Python wrapper using the extracted executable (\texttt{PaxosMain.d.byte}) created using Disel.

\begin{multicols*}{2}
\begin{lstlisting}[style=SourceCodeListing, language=Python]
#!/usr/bin/env python3

import subprocess
import re
import time


proposer1 = "(./PaxosMain.d.byte -me 1 -mode proposer 1 127.0.0.1 8000 2 127.0.0.1 8001 3 127.0.0.1 8002 4 127.0.0.1 8003 5 127.0.0.1 8004 &) > proposer1.log 2>&1"
proposer2 = "(./PaxosMain.d.byte -me 2 -mode proposer 1 127.0.0.1 8000 2 127.0.0.1 8001 3 127.0.0.1 8002 4 127.0.0.1 8003 5 127.0.0.1 8004 &) > proposer2.log 2>&1"
acceptor1 = "(./PaxosMain.d.byte -me 3 -mode acceptor 1 127.0.0.1 8000 2 127.0.0.1 8001 3 127.0.0.1 8002 4 127.0.0.1 8003 5 127.0.0.1 8004 &) > acceptor1.log 2>&1"
acceptor2 = "(./PaxosMain.d.byte -me 4 -mode acceptor 1 127.0.0.1 8000 2 127.0.0.1 8001 3 127.0.0.1 8002 4 127.0.0.1 8003 5 127.0.0.1 8004 &) > acceptor2.log 2>&1"
acceptor3 = "(./PaxosMain.d.byte -me 5 -mode acceptor 1 127.0.0.1 8000 2 127.0.0.1 8001 3 127.0.0.1 8002 4 127.0.0.1 8003 5 127.0.0.1 8004 &) > acceptor3.log 2>&1"

a1 = subprocess.Popen(acceptor1, shell=True)
a2 = subprocess.Popen(acceptor2, shell=True)
a3 = subprocess.Popen(acceptor3, shell=True)
p1 = subprocess.Popen(proposer1, shell=True)
p2 = subprocess.Popen(proposer2, shell=True)

consensus_achieved = True
consensus_value = None

exit_codes = [p.wait() for p in [a1, a2, a3]]
print(exit_codes)

# wait for processes to write to files
time.sleep(2)

for i in range(1, 4):
    line = subprocess.check_output(['tail', '-1', "acceptor{0}.log".format(i)])
    line = line.decode("utf-8")
    print(line.rstrip())

    pattern = r"got msg in protocol (\d) with tag = (\d), contents = \[(\d); (\d)\]"
    match = re.match(pattern, line)

    if not match or (consensus_value and consensus_value != match.group(4)):
        print("Error in acceptor %d" % i)
        consensus_achieved = False
        break
    else:
        print("Acceptor %d accepted %s" % (i, match.group(4)))
        consensus_value = match.group(4)

print("=" * 40)

if consensus_achieved:
    print("\nConsesus achieved on value: %s" % consensus_value)
else:
    print("\nConsesus not achieved")
\end{lstlisting}
\end{multicols*}

\section{shims/PaxosMain.ml}
\begin{multicols*}{2}
\begin{lstlisting}[style=SourceCodeListing]
open Datatypes

open Util
open Shim
open Unix

type mode = Proposer | Acceptor

let mode : mode option ref = ref None
let server_name : Datatypes.nat option ref = ref None
let me : Datatypes.nat option ref = ref None
let nodes : (Datatypes.nat * (string * int)) list ref = ref []

let usage msg =
  print_endline msg;
  Printf.printf "%s usage:\n" Sys.argv.(0);
  Printf.printf "    %s [OPTIONS] <CLUSTER>\n" (Array.get Sys.argv 0);
  print_endline "where:";
  print_endline "    CLUSTER   is a list of triples of ID IP_ADDR PORT,";
  print_endline "              giving all the nodes in the system";
  print_newline ();
  print_endline "Options are as follows:";
  print_endline "    -me <NAME>           the identity of this node (required)";
  print_endline "    -mode <MODE>         whether this node is the proposer or acceptor (required)";
  print_endline "    -proposer <NAME>  the identity of the proposer (required if mode=client)";
  exit 1


let rec parse_args = function
  | [] -> ()
  | "-mode" :: "acceptor" :: args ->
      begin
        mode := Some Acceptor;
        parse_args args
      end
  | "-mode" :: "proposer" :: args ->
      begin
        mode := Some Proposer;
        parse_args args
      end
  | "-me" :: name :: args ->
     begin
       me := Some (nat_of_string name);
       parse_args args
     end
  | "-proposer" :: name :: args ->
     begin
       server_name := Some (nat_of_string name);
       parse_args args
     end
  | name :: ip :: port :: args -> begin
      nodes := (nat_of_string name, (ip, int_of_string port)) :: !nodes;
      parse_args args
    end
  | arg :: args ->
     usage ("Unknown argument " ^ arg)


let main () =
  parse_args (List.tl (Array.to_list Sys.argv));
  match !mode, !me with
  | Some mode, Some me ->
    begin
      Shim.setup { nodes = !nodes; me = me; st = SimplePaxosApp.init_state };
      match mode with
      | Acceptor ->
        begin match int_of_nat me with
        | 3 ->
          begin
            try
              SimplePaxosApp.a_runner1 ()
            with _ -> print_endline "acceptor 1 exiting."
          end
        | 4 ->
          begin
            try
              SimplePaxosApp.a_runner2 ()
            with _ -> print_endline "acceptor 2 exiting."
          end
        | 5 ->
          begin
            try
              SimplePaxosApp.a_runner3 ()
            with _ -> print_endline "acceptor 3 exiting."
          end
        | n -> usage ("unknown acceptor name " ^ string_of_int n)
        end
      | Proposer ->
        begin match int_of_nat me with
        | 1 ->
          begin
            try
              SimplePaxosApp.p_runner1 ()
            with _ -> print_endline "A acceptor closed its connection, proposer 1 exiting."
          end
        | 2 ->
          begin
            try
              SimplePaxosApp.p_runner2 ()
            with _ -> print_endline "A acceptor closed its connection, proposer 2 exiting."
          end
        | n -> usage ("unknown proposer name " ^ string_of_int n)
        end
    end
  | _, _ -> usage "-mode and -me must be given"

let _ = main ()
\end{lstlisting}
\end{multicols*}

\iffalse
\section{PaxosProtocol.v}
\begin{multicols*}{2}
\begin{lstlisting}[style=SourceCodeListing]
From mathcomp.ssreflect
Require Import ssreflect ssrbool ssrnat eqtype ssrfun seq.
From mathcomp
Require Import path.
Require Import Eqdep.
Require Import Relation_Operators.
From DiSeL.Heaps
Require Import pred prelude idynamic ordtype finmap pcm unionmap heap coding domain.
From DiSeL.Core
Require Import Freshness State EqTypeX DepMaps Protocols Worlds NetworkSem Rely.
From DiSeL.Core
Require Import Actions Injection Process Always HoareTriples InferenceRules.

Set Implicit Arguments.
Unset Strict Implicit.
Import Prenex Implicits.


Module PaxosProtocol.


Module States.


Definition nid := nat.

(* seq of two elements p_no, p_val *)
Definition proposal := seq nat.https://github.com/anirudhpillai/disel
(* Promise -> seq (node * promise/nack * accepted_proposal) *)
Definition promises := seq (nid * bool * proposal).

(* States of the nodes *)
Inductive RoleState :=
(* Proposer States *)
(* Initialised with a proposal (p_no * p_val) *)
| PInit of proposal
(* Sent prepare message to some Acceptors at a current stage *)
(* seq nid holds nodes which were sent the message *)
| PSentPrep of seq nid & proposal
(* Received promises/NACKs from Acceptors *)
| PWaitPrepResp of promises & proposal
(* Send AcceptRequest *)
| PSentAccReq of seq nid & proposal
(* Finished executing after sending AccReq or not receiving majority*)
| PAbort
(* Acceptor states *)
| AInit
(* Holds the highest number promised in the proposal *)
| APromised of proposal
(* Holds the highest number proposal accepted *)
| AAccepted of proposal.

(* Pointer to the state *)
Definition st := ptr_nat 1.

(* Pairing with the current stage of type nat *)
Definition StateT := (nat * RoleState)%type.


End States.

Import States.

Section PaxosProtocol.

(* Proposer nodes *)
Variable proposers : seq nid.
(* Acceptor nodes *)
Variable acceptors : seq nid.
(* Involved nodes *)
Definition nodes := proposers ++ acceptors.


Definition prepare_req : nat := 0.
Definition promise_resp : nat := 1.
Definition nack_resp : nat := 2.
Definition accept_req : nat := 3.


Definition ttag := nat.

Definition tags : seq ttag :=
  [:: prepare_req; promise_resp; nack_resp; accept_req].


(*** Defining Coherence ***)

(**
Coherence predicate defines the shape of the statelet.
i.e. it components of the local state and message soup properties.
Therefore, we need to write coherence functions for both the
local state and message soup and then combine them to create PaxosCoh.
 *)


(** localCoh constraints the local state of each node. **)
Definition localCoh (n : nid) : Pred heap :=
  [Pred h | valid h /\ exists (s : StateT), h = st :-> s].

Definition tagFromNode (t : nat) : bool :=
  (t \in [:: prepare_req; accept_req; promise_resp; nack_resp]).

Definition msgFromNode (tms : TaggedMessage) (y : nat) : Prop :=
  let: body := tms_cont tms in exists data, body = y :: data.

Definition cohMsg (ms: msg TaggedMessage) (y : nat) : Prop := True.

(** Coherence for the message soup. *)
Definition soupCoh : Pred soup :=
  [Pred s | valid s /\
            forall m ms, find m s = Some ms -> exists y, cohMsg ms y].

Definition paxos_coh d : Prop :=
  let: dl := dstate d in
  let: ds := dsoup d in
  [/\ soupCoh ds, dom dl =i nodes,
   valid dl &
   forall n, n \in nodes -> localCoh n (getLocal n d)].

(* Axioms of the coherence predicate *)
Lemma l1 d: paxos_coh d -> valid (dstate d).
Proof. by case. Qed.

Lemma l2 d: paxos_coh d -> valid (dsoup d).
Proof. by case; case. Qed.

Lemma l3 d: paxos_coh d -> dom (dstate d) =i nodes.
Proof. by case. Qed.

(* Wrapping up the coherence predicate *)
Definition PaxosCoh := CohPred (CohPredMixin l1 l2 l3).

Lemma send_soupCoh d m :
    soupCoh (dsoup d) -> (exists y, cohMsg m y) -> soupCoh (post_msg (dsoup d) m).1.
Proof.
  move=>[H1 H2][y]Cm; split=>[|i ms/=]; first by rewrite valid_fresh.
  rewrite findUnL; last by rewrite valid_fresh.
  case: ifP=>E; first by move/H2.
    by move/um_findPt_inv=>[Z G]; subst i m; exists y.
Qed.

Lemma trans_updDom this d s :
  this \in nodes -> PaxosCoh d -> dom (upd this s (dstate d)) =i nodes.
Proof.
  move=>D C z; rewrite -(cohDom C) domU inE/=.
    by case: ifP=>///eqP->{z}; rewrite (cohDom C) D; apply: cohVl C.
Qed.

Lemma cohSt n d (C : PaxosCoh d) s:
  find st (getLocal n d) = Some s ->
  idyn_tp s = StateT.
Proof.
case: (C)=> _ _ _ G; case H: (n \in nodes).
- by move: (G _ H); case=>V'[s']->; rewrite hfindPt//=; case=><-.
rewrite /getLocal; rewrite -(cohDom C) in H.
by case: dom_find H=>//->; rewrite find0E.
Qed.

Definition getSt n d (C : PaxosCoh d) : StateT :=
  match find st (getLocal n d) as f return _ = f -> _ with
  | Some v => fun epf => icoerce id (idyn_val v) (cohSt C epf)
  | _ => fun epf => (0, AInit)
  end (erefl _).

Lemma locCn n d (C : PaxosCoh d):
  n \in nodes ->
  valid (getLocal n d) /\
  exists (s : StateT), getLocal n d = st :-> s.
Proof.
by case: C=>_ _ _ /(_ n)G; move: G; rewrite /localCoh/=.
Qed.

Lemma getStK n d (C : PaxosCoh d)  s :
  getLocal n d = st :-> s -> getSt n C = s.
Proof.
move=>E; rewrite /getSt/=.
move: (cohSt C); rewrite !E/==>H.
by apply: ieqc.
Qed.

Lemma getStE n i j C C' (pf : n \in nodes) :
  getLocal n j = getLocal n i -> @getSt n j C' = @getSt n i C.
Proof.
case: {-1}(C)=>_ _ _/(_ _ pf).
move=>[V][s][E]E'.
rewrite (getStK C E).
by apply: getStK; rewrite E'.
Qed.

Lemma getStE' l i j pf pf' n:
  getLocal n (getStatelet j l) = getLocal n (getStatelet i l) ->
  @getSt n (getStatelet j l) pf' = @getSt n (getStatelet i l) pf.
Proof.
Admitted.

(*** State Transitions ***)
Definition fst' (tup: (nat * bool * proposal)%type): nat :=
  match tup with
  | (x, b, props) => x
  end.

Definition snd' (tup: (nat * bool * proposal)%type): bool :=
  match tup with
  | (x, b, props) => b
  end.

Fixpoint find_highest_numbered_promise (max_so_far: proposal) (xs: promises):
  proposal :=
  match xs with
  | cons (_, check, p) rest =>
    if check && (head 0 p > head 0 max_so_far)
    then find_highest_numbered_promise p rest
    else find_highest_numbered_promise max_so_far rest
  | _ => max_so_far
  end.

(* Choose value of highest numbered proposal received from acceptors *)
Fixpoint create_proposal_for_acc_req (xs: promises) (p: proposal): proposal :=
  (* All promises received *)
  if (all (fun i => i) (map snd' xs))
  then
    let: max_proposal := find_highest_numbered_promise [:: 0; 0] xs in
    if perm_eq max_proposal [:: 0; 0]
    then p
    else [:: (head 0 p); (last 0 max_proposal)]
  else [:: 0; 0].

(**
Step Functions

Step function dictactes how the state of the node changes
after performing the send/receive transitions.
*)

(**
Send Transitions:
- Proposer: sPrep, sAccReq
- Acceptor: sPromise, sNack
*)

(* Changes in the Node state triggered upon send *)
Definition step_send (s: StateT) (to : nid) (p: proposal): StateT :=
    let: (e, rs) := s in
    match rs with
    | PInit p' =>
      if acceptors == [:: to] (* if only one acceptor *)
      then (e, PWaitPrepResp [::] p')
      else (e, PSentPrep [:: to] p')
    | PSentPrep tos p' =>
      if perm_eq (to :: tos) acceptors
      (* If all prepare reqs sent *)
      then (e, PWaitPrepResp [::] p') (* switch to the receiving state *)
      else (e, PSentPrep (to :: tos) p') (* Keep sending *)
    | PSentAccReq tos p' =>
      if perm_eq (to :: tos) acceptors
      (* If all accept reqs sent *)
      then (e, PAbort)
      else (e, PSentAccReq (to :: tos) p') (* Keep sending *)
    | _ => (e, rs)
    end.

(**
Receive Transitions:
- Proposer: rPromise, rNack
- Acceptor: rPrep, rAccReq
 *)

Definition payload := proposal.

(* Changes in the Node state triggered upon receive *)
Definition step_recv (s : StateT) (from : nid) (mtag : ttag) (mbody : payload):
  StateT :=
  let: (e, rs) := s in
  let: p_no := head 0 mbody in
  let: p_val := last 0 mbody in
  match rs with
  | PWaitPrepResp recv_promises p' =>
    if (from \in (map fst' recv_promises)) (* If duplicate *)
    then s (* then ignore *)
    else if mtag == nack_resp
         then (e, PAbort) (* Abort if we see nack *)
         else
           let: r_promises := (from, mtag == promise_resp, mbody) :: recv_promises in
           (* if all promises received, we know we don't have nacks *)
           if (perm_eq (map fst' r_promises) acceptors)
           then let: new_p := create_proposal_for_acc_req r_promises p' in
                (e, PSentAccReq [::] new_p)
           else (e, PWaitPrepResp r_promises p') (* keep waiting for promises *)
  | AInit => (* Promise/Accept first received transition *)
    if mtag == prepare_req
    then (e, APromised mbody)
    else (e, AAccepted mbody)
  | APromised p' =>
      let: curr_p_no := head 0 p' in
      let: curr_p_val := last 0 p' in
      if mtag == prepare_req
      then if p_no > curr_p_no (* If received higher number *)
           (* Update promised number by storing new proposal *)
           then (e, APromised mbody)
           else (e, APromised p')
      else (* It's an accept request *)
           if p_no > curr_p_no (* If received higher number *)
           then (e, AAccepted mbody) (* Accept the proposal *)
           else (e, APromised p') (* we'll send nack *)
  | _ => s
  end.

(*
There should be 4 send-transitions for the node:
- send-prepare-request
- send-accept-request
- send-promise
- send-nack

There should be 4 receive-transitions for the node:
- receive-promise
- receive-nack
- receive-prepare-request
- receive-accept-request
 *)

Section GenericSendTransitions.

Notation coh := PaxosCoh.

Definition Hin this to := (this \in nodes /\ to \in nodes).
Definition mkLocal {T} (sl : T) := st :-> sl.
Check mkLocal.

Variable ptag : ttag.

(* Precondition -- this is the way one can define multiple send-transitions *)
Variable prec : StateT -> payload -> Prop.

Hypothesis node_prec_safe :
  forall this to s m,
    Hin this to -> prec s m -> cohMsg (Msg (TMsg ptag m) this to true) s.1.

(* Making sure that the precondition is legit *)
Lemma this_in this to : Hin this to -> this \in nodes.
Proof.
  by case.
Qed.

Definition node_safe (this n : nid)
           (d : dstatelet) (msg : payload) :=
  Hin this n /\
  exists (Hp : Hin this n) (C : coh d), prec (getSt this C) msg.

Lemma node_safe_coh this to d m : node_safe this to d m -> coh d.
Proof.
  case.
  move => in_nodes.
  case.
  move => in_nodes2.
  case => coh_d H_coh_d => //.
Qed.

Lemma node_safe_in this to d m : node_safe this to d m ->
                               this \in nodes /\ to \in nodes.
Proof.
  case.
  move => Hin e_clause => //.
Qed.

Definition node_step (this to : nid) (d : dstatelet)
           (msg : payload)
           (pf : node_safe this to d msg) :=
  let C := node_safe_coh pf in
  let s := getSt this C in
  Some (mkLocal (step_send s to msg)).

Lemma node_step_coh : s_step_coh_t coh ptag node_step.
Proof.
  move=>this to d msg pf h[]->{h}.
  have C : (coh d) by case: pf=>?[?][].
  split=>/=.
  - apply: send_soupCoh; first by case:(node_safe_coh pf).
    exists (getSt this C).1.
    case: (pf)=> H[C']P /=. move: (conj H P)=>pf'.
    move: P.
    case => H2 P.
    move: (node_prec_safe H P). rewrite (proof_irrelevance C H2)/=. done.
  - by apply: trans_updDom=>//; case: (node_safe_in pf).
  - by rewrite validU; apply: cohVl C.
  move=>n Ni. rewrite /localCoh/=.
  rewrite /getLocal/=findU; case: ifP=>B; last by case: C=>_ _ _/(_ n Ni).
  move/eqP: B=>Z; subst n=>/=.
  rewrite (cohVl C)/=; split=>//.
  move: (step_send _ _ _)=>ps.
  rewrite ?hvalidPtUn//; last by eexists _.
Qed.

Lemma node_safe_def this to d msg :
      node_safe this to d msg <->
      exists b pf, @node_step this to d msg pf = Some b.
Proof.
  split=>[pf/=|]; last by case=>?[].
  set b := let C := node_safe_coh pf in
         let s := getSt this C in
         mkLocal (step_send s to msg).
  by exists b, pf.
Qed.

Definition node_send_trans :=
  SendTrans node_safe_coh node_safe_in node_safe_def node_step_coh.

End GenericSendTransitions.

Section SendTransitions.

(* Send prepare request transition *)
Definition send_prepare_req_prec (p: StateT) (m: payload) :=
  (exists n psal, p = (n, PInit psal)) \/
  (exists n tos psal, p = (n, PSentPrep tos psal)).

Program Definition send_prepare_req_trans : send_trans PaxosCoh :=
  @node_send_trans prepare_req send_prepare_req_prec _.
Next Obligation.
  by rewrite /cohMsg.
Qed.

(* Send accept request transition *)
Definition send_accept_req_prec (p: StateT) (m: payload) :=
  (exists n psal, p = (n, PSentAccReq [::] psal)).

Program Definition send_accept_req_trans : send_trans PaxosCoh :=
  @node_send_trans accept_req send_accept_req_prec _.
Next Obligation.
  by rewrite /cohMsg.
Qed.

(* Send promise response transition *)
Definition send_promise_resp_prec (p: StateT) (m: payload) :=
  (exists n, p = (n, AInit)) \/ (exists n psal, p = (n, APromised psal)).

Program Definition send_promise_resp_trans : send_trans PaxosCoh :=
  @node_send_trans promise_resp send_promise_resp_prec _.
Next Obligation.
  by rewrite /cohMsg.
Qed.

(* Send nack response transition *)
Definition send_nack_resp_prec (p: StateT) (m: payload) :=
  exists n psal, p = (n, APromised psal).

Program Definition send_nack_resp_trans : send_trans PaxosCoh :=
  @node_send_trans nack_resp send_nack_resp_prec _.
Next Obligation.
  by rewrite /cohMsg.
Qed.

End SendTransitions.

Section GenericReceiveTransitions.

Notation coh := PaxosCoh.

Variable r_tag : ttag.
Variable r_wf : forall d, coh d -> nid -> nid -> pred payload.

Definition r_step : receive_step_t coh :=
  fun this (from : nid) (m : proposal) d (pf : coh d) (pt : this \in nodes) =>
    let s := getSt this pf in
    mkLocal (step_recv s from r_tag m).

Lemma r_step_coh : r_step_coh_t r_wf r_tag r_step.
Proof.
  move=>d from this m C pf tms D F Wf T/=.
  rewrite /r_step; case X: (this \in nodes); last first.
  admit.
Admitted.

Definition recv_trans := ReceiveTrans r_step_coh.

End GenericReceiveTransitions.

Section ReceiveTransitions.

Definition msg_wf d (_ : PaxosCoh d) (this from : nid) :=
  [pred p : payload | true].

(* Got prepare request *)
Definition receive_prepare_req_trans := recv_trans prepare_req msg_wf.

(* Got accept request *)
Definition receive_accept_req_trans := recv_trans accept_req msg_wf.

(* Got promise response *)
Definition receive_promise_resp_trans := recv_trans promise_resp msg_wf.

(* Got nack response *)
Definition receive_nack_resp_trans := recv_trans nack_resp msg_wf.

End ReceiveTransitions.


Section Protocol.

(* Putting it all together *)
Variable l : Label.

(* All send-transitions *)
Definition paxos_sends :=
  [::
     send_prepare_req_trans;
     send_accept_req_trans;
     send_promise_resp_trans;
     send_nack_resp_trans
  ].

(* All receive-transitions *)
Definition paxos_receives :=
  [::
     receive_prepare_req_trans;
     receive_accept_req_trans;
     receive_promise_resp_trans;
     receive_nack_resp_trans
  ].

Program Definition PaxosProtocol : protocol :=
  @Protocol _ l _ paxos_sends paxos_receives _ _.

End Protocol.
End PaxosProtocol.

Module Exports.
Section Exports.

Definition PaxosProtocol := PaxosProtocol.

Definition send_prepare_req_trans := send_prepare_req_trans.
Definition send_accept_req_trans := send_accept_req_trans.
Definition send_promise_resp_trans := send_promise_resp_trans.
Definition send_nack_resp_trans := send_nack_resp_trans.

Definition receive_prepare_req_trans := receive_prepare_req_trans.
Definition receive_accept_req_trans := receive_accept_req_trans.
Definition receive_promise_resp_trans := receive_promise_resp_trans.
Definition receive_nack_resp_trans := receive_nack_resp_trans.

(* Paxos Tags *)
Definition prepare_req := prepare_req.
Definition accept_req := accept_req.
Definition promise_resp := promise_resp.
Definition nack_resp := nack_resp.

(* Getter *)
Definition getSt := getSt.

End Exports.
End Exports.

End PaxosProtocol.

Export PaxosProtocol.States.
Export PaxosProtocol.Exports.
\end{lstlisting}
\end{multicols*}

\newpage

\section{PaxosProposer.v}
\begin{multicols*}{2}
\begin{lstlisting}[style=SourceCodeListing]
From mathcomp.ssreflect
Require Import ssreflect ssrbool ssrnat eqtype ssrfun seq.
From mathcomp
Require Import path.
Require Import Eqdep.
Require Import Relation_Operators.
From DiSeL.Heaps
Require Import pred prelude idynamic ordtype finmap pcm unionmap heap coding domain.
From DiSeL.Core
Require Import Freshness State EqTypeX DepMaps Protocols Worlds NetworkSem Rely.
From DiSeL.Core
Require Import Actions Injection Process Always HoareTriples InferenceRules.
From DiSeL.Core
Require Import InductiveInv While.
From DiSeL.Examples
Require Import PaxosProtocol.


Module PaxosProposer.
Section PaxosProposer.

Variable l : Label.
Variables (proposers: seq nid) (acceptors: seq nid).
Variable p: nid.

Hypothesis AcceptorsNonEmpty : acceptors != [::].

Check PaxosProtocol.

Definition paxos := PaxosProtocol proposers acceptors l.
Notation W := (mkWorld paxos).

Section ProposerImplementation.

(************** Atomic actions **************)

(* Two send-actions, e -- id of the current era *)
Program Definition send_prepare_req psal to :=
  act (@send_action_wrapper W paxos p l (prEq paxos)
       (send_prepare_req_trans proposers acceptors) _ psal to).
Next Obligation.
  by rewrite InE; do![left|right].
Qed.

Program Definition send_accept_req psal to :=
  act (@send_action_wrapper W paxos p l (prEq paxos)
       (send_accept_req_trans proposers acceptors) _ psal to).
Next Obligation.
  by rewrite !InE; right; left.
Qed.


(* Two receive-actions *)
Program Definition tryrecv_prepare_resp := act (@tryrecv_action_wrapper W p
      (* filter *)
      (fun k _ t b => (k == l) && ((t == promise_resp) || (t == nack_resp))) _).
Next Obligation.
  by case/andP: H=>/eqP->_; rewrite /ddom gen_domPt inE/=.
Qed.


(************** Proposer code **************)

(*** Reading internal state ***)
Implicit Arguments PaxosProtocol.PaxosCoh [proposers acceptors].
Notation coh := (@PaxosProtocol.PaxosCoh proposers acceptors).
Notation getS s := (getStatelet s l).
Notation loc i := (getLocal p (getStatelet i l)).

Export PaxosProtocol.

Program Definition read_round:
  {(s: StateT)}, DHT [p, W]
  (fun i => loc i = st :-> s,
   fun r m => loc m = st :-> s /\
              exists (pf : coh (getS m)), r = (getSt p pf).1) :=
  Do (act (@skip_action_wrapper W p l paxos (prEq paxos) _
                                (fun s pf => (getSt p pf).1))).
Next Obligation.
  apply: ghC.
  move => s st s_is_st s_in_coh_w.
  apply: act_rule => j R.
  split => [|r k m].
    by case: (rely_coh R).
  case=>/=H1[Cj]Z.
  subst j=>->R'.
  split; first by rewrite (rely_loc' l R') (rely_loc' _ R).
  case: (rely_coh R')=>_; case=>_ _ _ _/(_ l)=>/= pf; rewrite prEq in pf.
  exists pf; move: (rely_loc' l R') =>/sym E'.
  suff X: getSt p (Actions.safe_local (prEq paxos) H1) = getSt p pf by rewrite X.
  by apply: (getStE' pf _ E').
Qed.

(*******************************************)
(***   Sending out proposals in a loop   ***)
(*******************************************)

Definition send_prepare_req_loop_spec (e : nat) := forall to_send,
  {(pinit: proposal)}, DHT [p, W]
  (fun i =>
     loc i = st :-> (e, PInit pinit) /\
     (perm_eq acceptors to_send \/
     if to_send == [::]
     then loc i = st :-> (e, PWaitPrepResp [::] pinit)
     else exists (acptrs : seq nid),
         loc i = st :-> (e, PSentPrep acptrs pinit) /\
         perm_eq acceptors (acptrs ++ to_send)),
     fun r m => r = tt /\ loc m = st :-> (e, PWaitPrepResp [::] pinit)).

Program Definition send_prepare_req_loop e (psal: proposal):
  {(pinit: proposal)}, DHT [p, W]
  (fun i => loc i = st :-> (e, PInit pinit),
   fun r m => r = tt /\
              loc m = st :-> (e, PWaitPrepResp [::] pinit)) :=
  Do (ffix (fun (rec : send_prepare_req_loop_spec e) to_send =>
              Do (match to_send with
                  | to :: tos => send_prepare_req psal to ;; rec tos
                  | [::] => ret _ _ tt
                  end)) acceptors).
Next Obligation.
  apply: ghC => i1 p'.
  case=>[[E1 P1 C1]].
  case: (to_send)=>[|x xs]; last first.
  apply: step; apply: act_rule => j1 R1/=; split=>[|r k m[Sf]St R2].
  split=>//=; first by case: (rely_coh R1).
  rewrite /node_safe.
  split.
  split.
  admit. (* what does p \in nodes proposers acceptors mean? *)
  admit.
  split.
  admit. (* again Hin *)
  rewrite /send_prepare_req_prec.
  admit. (* Where do I get coherence from *)
  (* because of inE *)
  (* + rewrite /Actions.can_send /nodes inE eqxx andbC/=. *)
  (*   by rewrite -(cohD (proj2 (rely_coh R1)))/ddom gen_domPt inE/=. *)
  admit.
  by rewrite /Actions.filter_hooks um_filt0=>???/sym/find_some; rewrite dom0 inE.
  admit.
  admit.
Admitted.
Next Obligation.
  apply:ghC=>i lg E1 _; apply: (gh_ex (g:=lg)).
  apply: call_rule=>C1. split => //. by left. done.
Qed.

(*******************************************)
(*** Receiving responses to the proposal ***)
(*******************************************)

(* Ending condition *)
Definition rc_prepare_resp_cond (recv_promises : promises) :=
  ~~ perm_eq (map fst' recv_promises) acceptors.

(* Invariant relates the argument and the shape of the state *)
Definition rc_prepare_resp_inv (e : nat) (psal: proposal): cont (promises) :=
  fun acc i => loc i = st :-> (e, PWaitPrepResp acc psal).

Program Definition receive_prepare_resp_loop (e : nat):
  {(pinit : proposal)}, DHT [p, W]
  (fun i => loc i = st :-> (e, PWaitPrepResp [::] pinit),
   fun res m =>
       loc m = st :-> (e, PWaitPrepResp res pinit) /\
       (perm_eq (map fst' res) acceptors))
  :=
  Do _ (@while p W _ _ rc_prepare_resp_cond (rc_prepare_resp_inv e) _
        (fun recv_promises => Do _ (
           r <-- tryrecv_prepare_resp;
           match r with
           | Some (from, tg, body) =>
             if (from \in acceptors) && (from \notin (map fst' recv_promises))
             then ret _ _ ((from, tg == promise_resp, body) :: recv_promises)
             else ret _ _ recv_promises
           | None => ret _ _ recv_promises
           end
        )) [::]).
Next Obligation.
  by apply: with_spec x.
Defined.
Next Obligation.
  by move:H; rewrite /rc_prepare_resp_inv (rely_loc' _ H0).
Qed.
Next Obligation.
  move => i [psal] /= [H1 I1].
  apply: step.
  apply: act_rule=>j R1/=; split; first by case: (rely_coh R1).
  case=>[[[from tg] body] k m|k m]; last first.
  - case=>Sf []Cj[]H; last by case: H=>[?][?][?][?][?][?][].
    have E: k = j by case: H.
    move: H. subst k=>_ R2. apply: ret_rule.
    move => m' R3.
    move => x [] cond.
    rewrite /rc_prepare_resp_inv.
    by rewrite -(rely_loc' _ R1)-(rely_loc' _ R2)-(rely_loc' _ R3).
  case=>Sf []Cj[]=>[|[l'][mid][tms][from'][rt][pf][][E]Hin E1 Hw/=]; first by case.
  case/andP=>/eqP Z G->{k}[]Z1 Z2 Z3 R2; subst l' from' tg body.
  move: rt pf (coh_s (w:=W) l (s:=j) Cj) Hin R2 E1 Hw G E; rewrite prEq/=.
  move=>rt pf Cj' Hin R E1 Hw G E.
  have D: rt = receive_prepare_req_trans _ _.
  - case: Hin G=>/=; first by intuition.
  admit.
  admit.
Admitted.
Next Obligation.
  apply: ghC=>i pinit E1 C1.
  have Pre: rc_prepare_resp_inv e pinit [::] i by rewrite /rc_prepare_resp_inv/= E1.
  apply: call_rule'=>[|acc m]; first by exists pinit.
  case/(_ pinit Pre)=>/=H1 H2 Cm; split=>//;  by move/negbNE: H1.
Qed.

Definition read_res (st : StateT) :=
  let: (_, rs) := st in
  match rs with
  | PWaitPrepResp res _ => res
  | _ => [::]
  end.

(*******************************************)
(***    Sending accept requests          ***)
(*******************************************)

Definition send_accept_req_loop_spec (e : nat) psal := forall to_send,
  {(pinit: proposal)}, DHT [p, W]
  (fun i =>
     (exists res,
         [/\ loc i = st :-> (e, PWaitPrepResp res pinit),
         to_send = acceptors &
         perm_eq (map fst' res) acceptors]) \/
      if to_send == [::]
      then loc i = st :-> (e, PAbort)
      else exists (acptrs : seq nid),
        loc i = st :-> (e, PSentAccReq acptrs psal) /\
        perm_eq acceptors (acptrs ++ to_send),
   fun (r : unit) m => loc m = st :-> (e, PAbort)).
   (* Aborts after sending all accept requests *)

Program Definition send_accept_req_loop e psal: send_accept_req_loop_spec e psal :=
  fun to_send  =>
    Do (fix rec to_send :=
          (match to_send with
           | to :: tos => send_accept_req psal to ;; rec tos
           | [::] => ret _ _ tt
           end)) to_send.
Next Obligation.
  apply: ghC=>s1 psal' E1 C1; elim: to_send s1 E1 C1=>//=.
  - move=>s1; case; first by case=>?[]_ Z; rewrite -Z in (AcceptorsNonEmpty).
    move => E1 _. apply: ret_rule=>i2 R. by rewrite (rely_loc' _ R).
  move=>to tos Hi s1 H C1.
  apply: step; apply: act_rule=>s2 R2/=.
  have Pre: Actions.send_act_safe W (p:=paxos) p l
          (send_accept_req_trans proposers acceptors) [:: e] to s2.
  - split; [by case: (rely_coh R2) | | |]; last first.
    + by rewrite /Actions.filter_hooks um_filt0=>???/sym/find_some; rewrite dom0 inE.
    (* Because of inE *)
    (* + rewrite /Actions.can_send /nodes inE eqxx andbC/=. *)
    (*     by rewrite -(cohD (proj2 (rely_coh R2)))/ddom gen_domPt inE/=. *)
      admit.
    case: (proj2 (rely_coh R2))=>_ _ _ _/(_ l); rewrite prEq=>C; split.
    (* Because of inE *)
    (* + split=>//; case: H; first by case=>?[_]<-; rewrite inE eqxx. *)
    (*     by case=>ps[_]/perm_eq_mem->; rewrite mem_cat orbC inE eqxx.   *)
    admit.
  split.
  admit. (* Because of Hin *)
  admit. (* Don't know why this doesn't work *)


  (* Postcondition *)
  split => //.
  admit. (* Actions.send_act_safe *)
  move => body i3 i4[Sf]/=St R3.
  apply: Hi; last by case: (rely_coh R3).
  right; rewrite (rely_loc' _ R3).
  admit.
Admitted.

Program Definition send_accept_reqs e psal:
  {(pinit: proposal)}, DHT [p, W]
  (fun i => exists recv_promises,
         [/\ loc i = st :-> (e, PWaitPrepResp recv_promises pinit) &
          perm_eq (map fst' recv_promises) acceptors],
   fun (r : unit) m => loc m = st :-> (e, PAbort))
  := Do (send_accept_req_loop e psal acceptors).
Next Obligation.
  apply: ghC=>i lg[res][H1]H2 H3 C; apply: (gh_ex (g:=lg)).
  apply: call_rule=>//; first by move=>_; left; exists res.
  done.
Qed.

Lemma equal_rounds (left_state: StateT) (right_state: StateT):
  left_state = right_state -> fst left_state = fst right_state.
Proof.
  admit.
Admitted.

(*****************************************************)
(*      Full Proposer Implementation                 *)
(*****************************************************)
Search _ (?X :-> _ = ?X :-> _).
(* This is only ment to be run once for each proposer *)
Program Definition proposer_round (p_init: proposal):
  {(psal: proposal) (e : nat)}, DHT [p, W]
  (fun i => loc i = st :-> (e, PInit psal),
   fun (_: unit) m => loc m = st :-> (e, PAbort))
  :=
  Do (e <-- read_round;
      send_prepare_req_loop e p_init;;
      recv_promises <-- receive_prepare_resp_loop e;
      send_accept_reqs e (create_proposal_for_acc_req recv_promises p_init);;
      ret _ _ tt).
Next Obligation.
  move=>s0/=[psal [e]] E0. apply: step.
  apply: (gh_ex (g := (e, PInit psal))).
  apply: call_rule=>//e' s1 [E1][pf]->C1.
  (* rewrite !(getStP_K _ E1)=>{e'}. *)
  (* apply: step; apply: (gh_ex (g := psal)). *)
  (* apply: call_rule=>//_ s2[_]/=E2 C2. *)
  (* apply: step; apply: (gh_ex (g:=psal)). *)
  (* apply: call_rule=>//res s3/= [E3 H3] C3. *)
  (* - do![apply: step]; apply: (gh_ex (g:=psal)). *)
  (*   apply: call_rule =>_. exists res. split => //. *)

  (*   move => s4 E4 C4. *)
  (*   apply: ret_rule => s5 R5 psal' e' E0'. *)
  (*   rewrite E0 in E0'. *)

    (* rewrite <- (equal_rounds _ _ (hcancelPtV _ E0')). *)
    (* Found no subterm matching "(e', PInit psal').1" in the current goal. *)
Admitted.

End ProposerImplementation.
End PaxosProposer.

Module Exports.
Section Exports.

Definition proposer_round := proposer_round.

End Exports.
End Exports.

End PaxosProposer.

Export PaxosProposer.Exports.
\end{lstlisting}
\end{multicols*}

\newpage

\section{PaxosAcceptor.v}
\begin{multicols*}{2}
\begin{lstlisting}[style=SourceCodeListing]
From mathcomp.ssreflect
Require Import ssreflect ssrbool ssrnat eqtype ssrfun seq.
From mathcomp
Require Import path.
Require Import Eqdep.
Require Import Relation_Operators.
From DiSeL.Heaps
Require Import pred prelude idynamic ordtype finmap pcm unionmap heap coding domain.
From DiSeL.Core
Require Import Freshness State EqTypeX DepMaps Protocols Worlds NetworkSem Rely.
From DiSeL.Core
Require Import Actions Injection Process Always HoareTriples InferenceRules.
From DiSeL.Core
Require Import InductiveInv While.
From DiSeL.Examples
Require Import PaxosProtocol.


Module PaxosAcceptor.
Section PaxosAcceptor.

Variable l : Label.
Variables (proposers: seq nid) (acceptors: seq nid).

Variable a: nid.


Definition paxos := PaxosProtocol proposers acceptors l.
Notation W := (mkWorld paxos).

Section AcceptorImplementation.

(************** Atomic actions **************)
(* Two send-actions, e -- id of the current era *)
Program Definition send_promise_resp psal to :=
  act (@send_action_wrapper W paxos a l (prEq paxos)
       (send_promise_resp_trans proposers acceptors) _ psal to).
Next Obligation.
  rewrite !InE. right. right. left. done.
Qed.

Program Definition send_nack_resp psal to :=
  act (@send_action_wrapper W paxos a l (prEq paxos)
       (send_nack_resp_trans proposers acceptors) _ psal to).
Next Obligation.
  rewrite !InE. right. right. right. done.
Qed.


(* Two receive-actions *)
Program Definition tryrecv_prepare_req := act (@tryrecv_action_wrapper W a
      (fun k _ t b => (k == l) && (t == prepare_req)) _).
Next Obligation.
  by case/andP: H=>/eqP->_; rewrite /ddom gen_domPt inE/=.
Qed.

Program Definition tryrecv_accept_req :=
  act (@tryrecv_action_wrapper W a
  (fun k _ t b => (k == l) && (t == accept_req)) _).
Next Obligation.
  by case/andP: H=>/eqP->_; rewrite /ddom gen_domPt inE/=.
Qed.


(************** Acceptor code **************)

(*** Reading internal state ***)
Implicit Arguments PaxosProtocol.PaxosCoh [proposers acceptors].
Notation coh := (@PaxosProtocol.PaxosCoh proposers acceptors).
Notation getS s := (getStatelet s l).
Notation loc i := (getLocal a (getStatelet i l)).

Export PaxosProtocol.

Program Definition read_round:
  {(s: StateT)}, DHT [a, W]
  (fun i => loc i = st :-> s,
   fun r m => loc m = st :-> s /\
              exists (pf : coh (getS m)), r = (getSt a pf).1) :=
  Do (act (@skip_action_wrapper W a l paxos (prEq paxos) _
                                (fun s pf => (getSt a pf).1))).
Next Obligation.
  apply: ghC.
  move => s st s_is_st s_in_coh_w.
  apply: act_rule => j R.
  split => [|r k m].
    by case: (rely_coh R).
  case=>/=H1[Cj]Z.
  subst j=>->R'.
  split; first by rewrite (rely_loc' l R') (rely_loc' _ R).
  case: (rely_coh R')=>_; case=>_ _ _ _/(_ l)=>/= pf; rewrite prEq in pf.
  exists pf; move: (rely_loc' l R') =>/sym E'.
  suff X: getSt a (Actions.safe_local (prEq paxos) H1) = getSt a pf by rewrite X.
    by apply: (getStE' pf _ E').
Qed.

Program Definition read_state:
  {(s: StateT)}, DHT [a, W]
  (fun i => loc i = st :-> s,
   fun r m => loc m = st :-> s /\
              exists (pf : coh (getS m)), r = (getSt a pf).2) :=
  Do (act (@skip_action_wrapper W a l paxos (prEq paxos) _
                                (fun s pf => (getSt a pf).2))).
Next Obligation.
  apply: ghC.
  move => s st s_is_st s_in_coh_w.
  apply: act_rule => j R.
  split => [|r k m].
    by case: (rely_coh R).
  case=>/=H1[Cj]Z.
  subst j=>->R'.
  split; first by rewrite (rely_loc' l R') (rely_loc' _ R).
  case: (rely_coh R')=>_; case=>_ _ _ _/(_ l)=>/= pf; rewrite prEq in pf.
  exists pf; move: (rely_loc' l R') =>/sym E'.
  suff X: getSt a (Actions.safe_local (prEq paxos) H1) = getSt a pf by rewrite X.
  by apply: (getStE' pf _ E').
Qed.

(* Step 1: Receive prepare req *)

(* Ending condition *)
Definition r_prepare_req_cond (res : option proposal) := res == None.

(* Invariant relates the argument and the shape of the state *)
Definition r_prepare_req_inv (e : nat) (pinit: proposal): cont (option proposal) :=
  fun res i =>
    if res is Some psal
    then loc i = st :-> (e, APromised psal)
    else loc i = st :-> (e, AInit).

(* Loops until it receives a prepare req *)
Program Definition receive_prepare_req_loop (e : nat):
  DHT [a, W]
  (fun i => loc i = st :-> (e, AInit),
   fun res m => exists psal, (res = Some psal) /\
       (loc m = st :-> (e, APromised psal)))
  :=
  Do _ (@while a W _ _ r_prepare_req_cond (r_prepare_req_inv e) _
        (fun _ => Do _ (
           r <-- tryrecv_prepare_req;
           match r with
           | Some (from, tg, body) => ret _ _ (Some body)
           | None => ret _ _ None
           end
        )) None).
Next Obligation. by apply: with_spec x. Defined.
Next Obligation. by move:H; rewrite /r_prepare_req_inv (rely_loc' _ H0). Qed.
Next Obligation.
  apply:ghC=>i1 psal[/eqP->{H}/=E1]C1; apply: step.
  apply: act_rule=>i2/=R1; split; first by case: (rely_coh R1).
  case=>[[[from e']d i3 i4]|i3 i4]; last first.
  - case=>S/=[]?; case; last by case=>?[?][?][?][?][?][].
    case=>_ _ Z; subst i3=>R2; apply: ret_rule=>i5 R4/=.
    by rewrite (rely_loc' _ R4) (rely_loc' _ R2)(rely_loc' _ R1).
  case=>Sf[]C2[]=>[|[l'][mid][tms][from'][rt][pf][][E]Hin E2 Hw/=]; first by case.
  case/andP=>/eqP Z G->[]Z1 Z2 Z3 R2; subst l' from' e' d.
  move: rt pf (coh_s (w:=W) l (s:=i2) C2) Hin R2 E2 Hw G E; rewrite prEq/=.
  move=>rt pf Cj' Hin R E2 Hw G E.
  have D: rt = receive_prepare_req_trans _ _.
  - move: Hin G; by do! [case=>/=; first by move=>->].
  subst rt=>{G}.
  have P1: valid (dstate (getS i2))
    by apply: (@cohVl _ coh); case: (Cj')=>P1 P2 P3 P4; split=>//=; done.
  have P2: valid i2 by apply: (cohS (proj2 (rely_coh R1))).
  have P3: l \in dom i2 by rewrite -(cohD (proj2 (rely_coh R1)))/ddom gen_domPt inE/=.

  apply: ret_rule=>//i5 R4.
  - rewrite /r_prepare_req_inv; rewrite (rely_loc' _ R4) (rely_loc' _ R) locE//=.
    rewrite /PaxosProtocol.r_step /=.
    rewrite -(rely_loc' _ R1) in E1.
    rewrite (getStK _ E1).
    by rewrite /step_recv /mkLocal.
Qed.
Next Obligation.
  move => i1/= E1.
  apply: (gh_ex (g:=([::0; 0]))).
  apply: call_rule => //r i2 [H1]H2 C2.
  rewrite /r_prepare_req_cond/r_prepare_req_inv in H1 H2.
    by case: r H1 H2 => //p _; exists p.
Qed.

(* Finds the promised number from current state *)
Definition read_promised_number (rs: RoleState): nat :=
  match rs with
  | APromised psal => head 0 psal
  | _ => 0
  end.

Definition read_promised_value (rs: RoleState): nat :=
  match rs with
  | APromised psal => last 0 psal
  | _ => 0
  end.

(* Step 2: Respond promise or nack to the proposer *)
Program Definition resp_to_prepare_req (e: nat) (send_promise: bool) (prepare_no: nat)
  (promised_no: nat) (promised_val: nat):
  {(pif: (proposal * proposal))}, DHT [a, W]
   (fun i => loc i = st :-> (e, APromised pif.1) \/ loc i = st :-> (e, AInit),
    fun (_ : seq nat) m =>
      if send_promise
      then loc m = st :-> (e, APromised pif.2)
      else loc m = st :-> (e, APromised pif.1))
  := Do (rs <-- read_state;
         if send_promise
         then send_promise_resp [:: promised_no; promised_val] prepare_no
         else send_nack_resp [:: 0; 0] prepare_no).
Next Obligation.
  apply:ghC=>i [pinit pfinal]E1 C1.
  have Pre: forall i2 proposer_id, network_rely W a i i2 ->
          Actions.send_act_safe W (p:=paxos) a l
          (if send_promise
           then send_promise_resp_trans proposers acceptors
           else send_nack_resp_trans proposers acceptors)
          [:: e] proposer_id i2.
  - move=>i2 pid R1.
    split; first by case: (rely_coh R1).
    case: (proj2 (rely_coh R1))=>_ _ _ _/(_ l); rewrite (prEq paxos)=>C.
    case: send_promise.
    split => //.
    admit.
    rewrite /send_promise_resp_prec.
    rewrite -(rely_loc' _ R1) in E1.
    (* Need inE to work *)
    (* + rewrite /Actions.can_send /nodes inE/= mem_cat Hpin orbC. *)
    (* by rewrite -(cohD (proj2 (rely_coh R1)))/ddom gen_domPt inE/= eqxx. *)

    (* by rewrite /Actions.filter_hooks um_filt0=>???/sym/find_some; rewrite dom0 inE. *)
Admitted.


(* Ending condition *)
Definition r_acc_req_cond (res : option proposal) := res == None.

(* Invariant relates the argument and the shape of the state *)
Definition r_acc_req_inv (e : nat) (promised: bool) (promised_psal: proposal) (received_psal: proposal): cont (option proposal) :=
  fun res i =>
    if res is Some received_psal
    then if promised
           then if head 0 promised_psal < head 0 received_psal
             then loc i = st :-> (e, AAccepted received_psal)
             else loc i = st :-> (e, APromised promised_psal)
         else loc i = st :-> (e, AAccepted received_psal)
    else if promised
         then loc i = st :-> (e, APromised promised_psal)
         else loc i = st :-> (e, AInit).

(* Loops until it receives a accept req *)
Program Definition receive_acc_req_loop (e : nat) (promised: bool) (promised_psal: proposal):
  {(pinit: proposal)}, DHT [a, W]
   (fun i => if promised
             then loc i = st :-> (e, APromised promised_psal)
             else loc i = st :-> (e, AInit),
  fun res m => exists psal, res = Some psal /\ (
    loc m = st :-> (e, APromised promised_psal) \/
    loc m = st :-> (e, AAccepted psal)
  )) :=
  Do _ (@while a W _ _ r_acc_req_cond (r_acc_req_inv e promised promised_psal) _
        (fun _ => Do _ (
           r <-- tryrecv_accept_req;
           match r with
           | Some (from, tg, body) => ret _ _ (Some body)
           | None => ret _ _ None
           end
        )) None).
Next Obligation. by apply: (with_spec x). Defined.
Next Obligation. by move:H; rewrite /r_acc_req_inv (rely_loc' _ H0). Qed.
Next Obligation.
  apply:ghC=>i1 psal [/eqP->{H}/=E1]C1. apply: step.
  apply: act_rule=>i2/=R1; split; first by case: (rely_coh R1).
  case=>[[[from e']v i3 i4]|i3 i4]; last first.
  - case=>S/=[]?; case; last by case=>?[?][?][?][?][?][].
    case=>_ _ Z; subst i3=>R2; apply: ret_rule=>i5 R4/=.
    rewrite /r_acc_req_inv/= in E1 *.
    by rewrite (rely_loc' _ R4) (rely_loc' _ R2) (rely_loc' _ R1).
  case=>Sf[]C2[]=>[|[l'][mid][tms][from'][rt][pf][][E]Hin E2 Hw/=]; first by case.
  case/andP=>/eqP Z G->[]Z1 Z2 Z3 R2; subst l' from' e' v.
  move: rt pf (coh_s (w:=W) l (s:=i2) C2) Hin R2 E2 Hw G E; rewrite prEq/=.
  move=>rt pf Cj' Hin R E2 Hw G E.
  have P1: valid (dstate (getS i2))
    by apply: (@cohVl _ coh); case: (Cj')=>P1 P2 P3 P4; split=>//=; done.
  have P2: valid i2 by apply: (cohS (proj2 (rely_coh R1))).
  have P3: l \in dom i2 by rewrite-(cohD (proj2 (rely_coh R1)))/ddom gen_domPt inE/=.
  have D: rt = receive_accept_req_trans _ _.
  -  by move: Hin G; clear E1; do![case=>/=; first by move=>->].
  apply: ret_rule=>//i5 R4.
  - rewrite /r_acc_req_inv (rely_loc' _ R4) (rely_loc' _ R) locE//=.
    case: (promised) E1 E2 D G=>/=E1 E2; case=>Z {E2}; subst rt=>//= _;
    rewrite -(rely_loc' _ R1) in E1;
    rewrite /r_step (getStK _ E1) /step_recv/= /mkLocal.
    case: ifP=>G1; case: ifP => //; rewrite G1 => //.
    done.
Qed.
Next Obligation.
  apply: ghC=>i1 psal E1 C1/=.
  apply: (gh_ex (g := psal)); apply: call_rule => //r i2 [H1]H2 C2.
  rewrite /r_acc_req_cond/r_acc_req_inv in H1 H2; case: r H1 H2=>//b _ i2_AA.
  exists b.
  split => //.
  move: i2_AA.
  case: ifP=>G1 => //.
  case: ifP=>G2.
  by right.
  by left.
  by right.
Qed.

(* Using resp_to_prepare_req 0 as a 'do nothing' transition for now.
As 0 will never be > 0 so the acceptor won't send a promise *)
Program Definition acceptor_round:
  {(ep: nat * proposal)}, DHT [a, W]
  (fun i => loc i = st :-> (ep.1, AInit),
   fun (_: unit) m => (
     loc m = st :-> (ep.1, AInit) \/
     loc m = st :-> (ep.1, APromised ep.2) \/
     loc m = st :-> (ep.1, AAccepted ep.2)
  )) :=
     Do _ (e <-- read_round;
          (* need to read state here as once it receives the message,
             it already changes state *)
          rs <-- read_state;
          msg <-- receive_prepare_req_loop e;
          (match msg with
           | Some body =>
             let: prepare_no := head 0 body in
             let: promised_no := read_promised_number rs in
             let: promised_val := read_promised_value rs in
             resp_to_prepare_req e (promised_no < prepare_no)
               prepare_no promised_no promised_val
           | _  => resp_to_prepare_req e false 0 0 0 (* results in sending nack *)
          end);;
         rs <-- read_state;
         (match rs with
         (* It's in promised state *)
         | APromised psal => receive_acc_req_loop e true psal
         (* It hasn't promised anything *)
         | _  => receive_acc_req_loop e false [:: 0; 0]
         end);;
         ret _ _ tt).
Next Obligation.
  apply: ghC => i1[e psal]/=E1 C1; apply: step.
  apply: (gh_ex (g:=(e, AInit))); apply: call_rule=>//e' i2 [E2][pf]->C2.
  apply: step.
  apply: (gh_ex (g:=(e, AInit))). apply: call_rule=>//??[?]??.
  apply: step.
  apply: call_rule; last first.
  move=>x? _ _. case: x=>a'. apply: step.
  apply: (gh_ex (g:=([::], [::]))). apply: call_rule; last first.
  move=>?? _ _. apply: step.
  apply: (gh_ex (g:=(e, AInit))). apply: call_rule; last first.
  move=>y ? _ _.
  case: y=>//.
Admitted.


End AcceptorImplementation.
End PaxosAcceptor.

Module Exports.
Section Exports.

Definition acceptor_round := acceptor_round.

End Exports.
End Exports.

End PaxosAcceptor.


Export PaxosAcceptor.Exports.
\end{lstlisting}
\end{multicols*}

\newpage

\section{SimplePaxosApp.v}
\begin{multicols*}{2}
\begin{lstlisting}[style=SourceCodeListing]
From mathcomp.ssreflect
Require Import ssreflect ssrbool ssrnat eqtype ssrfun seq.
From mathcomp
Require Import path.
Require Import Eqdep.
Require Import Relation_Operators.
From DiSeL.Heaps
Require Import pred prelude ordtype finmap pcm unionmap heap coding domain.
From DiSeL.Core
Require Import State Protocols Worlds NetworkSem Rely.
From DiSeL.Core
Require Import HoareTriples InferenceRules While.
From DiSeL.Examples
Require Import PaxosProtocol PaxosProposer PaxosAcceptor.
From DiSeL.Examples
Require PaxosInductiveInv.

Section SimplePaxosApp.

(*

A simple application to run on the shim implementation.

Check for [Run] tags to find the initial state and the code for the
proposers and the acceptors.

*)

Definition l := 0.
(* Proposer nodes *)
Definition p1 := 1.
Definition p2 := 2.

(* Acceptor nodes *)
Definition a1 := 3.
Definition a2 := 4.
Definition a3 := 5.

Definition proposers := [:: p1].
Definition acceptors := [::a1; a2; a3].

(* Necessary coherence facts *)
Fact AcceptorsNonEmpty : acceptors != [::]. Proof. by []. Qed.

(* Proposers *)
Definition proposer p (pf: acceptors != [::]) psal:=
  proposer_round l proposers acceptors p pf psal.

(* Acceptors *)
Program Definition acceptor a :=
  acceptor_round l proposers acceptors a.

(* Initial distributed state *)
Definition st_ptr := PaxosProtocol.States.st.

Definition init_heap_p psal:= st_ptr :-> (0, PInit psal).
Definition init_heap_a := st_ptr :-> (0, AInit).

Definition init_dstate :=
  p1 \\-> init_heap_p [:: 1; 1] \+
  a1 \\-> init_heap_a \+
  a2 \\-> init_heap_a \+
  a3 \\-> init_heap_a.

Lemma valid_init_dstate : valid init_dstate.
Proof.
  admit.
Admitted.

Notation init_dstatelet := (DStatelet init_dstate Unit).

(* [Run] Initial state to run *)
Definition init_state : state := l \\-> init_dstatelet.


(* Final Safety Facts *)
Notation W := (mkWorld (PaxosProposer.paxos l proposers acceptors)).

Lemma hook_complete_unit (c : context) : hook_complete (c, Unit).
Proof. by move=>????; rewrite dom0 inE. Qed.

Lemma hooks_consistent_unit (c : context) : hooks_consistent c Unit.
Proof. by move=>????; rewrite dom0 inE. Qed.

Lemma init_coh : init_state \In Coh W.
Proof. admit. Admitted.

(* [Run] Runnable proposer code *)
Program Definition run_proposer p (AcceptorsNonEmpty: acceptors != [::]) psal:
  DHT [p, _] (
    fun i => network_rely W p init_state i,
    fun _ m => exists (r : nat),
    getLocal p (getStatelet m l) = st :-> (r, PInit psal))
  := Do (with_inv (PaxosInductiveInv.ii _ _ _) (proposer p AcceptorsNonEmpty psal)).
Next Obligation.
  admit.
Admitted.
Check run_proposer.
(* [Run] Runnable acceptor code *)
Program Definition run_acceptor a:
  DHT [a, _] (
    fun i => network_rely W a init_state i,
    fun _ m => exists (r : nat),
    getLocal a (getStatelet m l) = st :-> (r, AInit))
  := Do (with_inv (PaxosInductiveInv.ii _ _ _) (acceptor a)).
Next Obligation.
  admit.
Admitted.

Variables (psal_1 psal_2 : proposal).
Variables (p_1 a_1 a_2 a_3: nat).
(* [Run] Runnable nodes *)
Program Definition run_proposer1 := run_proposer p1 AcceptorsNonEmpty psal_1.
Program Definition run_proposer2 := run_proposer p2 AcceptorsNonEmpty psal_2.
Program Definition run_acceptor1 := run_acceptor a_1.
Program Definition run_acceptor2 := run_acceptor a_2.
Program Definition run_acceptor3 := run_acceptor a_3.

End SimplePaxosApp.

(* [Run] Final programs to run with actual arguments supplied *)

Definition p_runner1 (u : unit) := run_proposer1 [:: 1; 1].
Definition p_runner2 (u : unit) := run_proposer1 [:: 2; 2].

Definition a_runner1 (u : unit) := run_acceptor1 a1.
Definition a_runner2 (u : unit) := run_acceptor2 a2.
Definition a_runner3 (u : unit) := run_acceptor3 a3.
\end{lstlisting}

\section{PaxosExtraction.v}
\begin{lstlisting}[style=SourceCodeListing]
From DiSeL.Core
Require Import DiSeLExtraction.
From DiSeL.Examples
Require Import SimplePaxosApp.

Cd "extraction".
  Cd "Paxos".
    Separate Extraction DepMaps.DepMaps.dmap init_state p_runner1 p_runner2 a_runner1 a_runner2 a_runner3.
  Cd "..".
Cd "..".
\end{lstlisting}
\end{multicols*}

\newpage

\section{PaxosInductiveInv.v}
\begin{multicols*}{2}
\begin{lstlisting}[style=SourceCodeListing]
From mathcomp.ssreflect
Require Import ssreflect ssrbool ssrnat eqtype ssrfun seq.
From mathcomp
Require Import path.
Require Import Eqdep.
Require Import Relation_Operators.
From DiSeL.Heaps
Require Import pred prelude idynamic ordtype finmap pcm unionmap.
From DiSeL.Heaps
Require Import heap coding domain.
From DiSeL.Core
Require Import Freshness State EqTypeX Protocols Worlds NetworkSem Rely.
From DiSeL.Core
Require Import Actions Injection Process Always HoareTriples InferenceRules.
From DiSeL.Core
Require Import InductiveInv StatePredicates.
From DiSeL.Examples
Require Import PaxosProtocol.


Set Implicit Arguments.
Unset Strict Implicit.
Import Prenex Implicits.

Section PaxosInductiveInv.

Variable l: Label.

Variable (proposers: seq nid) (acceptors: seq nid).

Definition paxos := PaxosProtocol proposers acceptors l.

(* Take the transitions *)
Notation sts := (snd_trans paxos).
Notation rts := (rcv_trans paxos).

Notation loc z d := (getLocal z d).

Definition role_state (d: dstatelet) (s: StateT) (n: nid): Prop :=
  loc n d = st :-> s.

(*****************************************************)

(* Phase Zero *)
Definition EverythingInit (d : dstatelet) (round : nat): Prop :=
  forall n, n \in (proposers ++ acceptors) -> (
    (exists psal, role_state d (round, PInit psal) n)
    \/ role_state d (round, AInit) n).


(* Phase One *)
Definition Phase1a (d: dstatelet) (round: nat) (n: nid): Prop :=
  if n \in acceptors
  then role_state d (round, AInit) n
  else (
    exists psal, (role_state d (round, PInit psal) n)
    \/ (exists sent_to, role_state d (round, PSentPrep psal sent_to) n)
    \/ (exists promises, role_state d (round, PWaitPrepResp promises psal) n)
  ).

Definition Phase1b (d: dstatelet) (round: nat) (n: nid): Prop :=
  if n \in acceptors
  then exists psal, role_state d (round, APromised psal) n
  else exists psal,
    exists promises, role_state d (round, PWaitPrepResp promises psal) n.

Definition Phase2a (d: dstatelet) (round: nat) (n: nid): Prop :=
  if n \in acceptors
  then (
    role_state d (round, PAbort) n
    \/ exists psal, (role_state d (round, APromised psal) n)
  )
  else exists psal,
    exists sent_to, role_state d (round, PSentAccReq sent_to psal) n.

Definition Phase2b (d: dstatelet) (round: nat) (n: nid): Prop :=
  if n \in acceptors
  then exists psal, (
    role_state d (round, AAccepted psal) n \/
    forall a, a \in acceptors -> role_state d (round, AAccepted psal) a
  )
  else role_state d (round, PAbort) n.

(* TODO: Fix the conditions. This Inv currently isn't checking for anything. *)
Definition Inv (d: dstatelet) :=
  exists round,
    EverythingInit d round
    \/ (forall n, n \in (proposers ++ acceptors) ->
         (Phase1a d round n \/ Phase1b d round n
          \/ Phase2a d round n \/ Phase2b d round n)).

(*********************************************************)

Notation Sinv := (@S_inv paxos (fun d _ => Inv d)).
Notation Rinv := (@R_inv paxos (fun d _ => Inv d)).
Notation coh d := (coh paxos d).
Notation PI := proof_irrelevance.

(*************************************************************)
(*                  Send-transitions                         *)
(*************************************************************)

Program Definition s1: Sinv (send_prepare_req_trans proposers acceptors).
Proof.
  move=>this to d msg S h/= Hi/=[]T G.
  case: (S)=>[][/eqP]Z H1[C]. exists to.

  (* Init state *)
  right.
  left.
  admit.
Admitted.

Program Definition s2: Sinv (send_accept_req_trans proposers acceptors).
Proof.
  move=>this to d msg S h/= Hi/=[]T G.
  case: (S)=>[][/eqP]Z H1[C]. exists to.

  (* Init state *)
  right.
  right.
  left.
  admit.
Admitted.

Program Definition s3: Sinv (send_promise_resp_trans proposers acceptors).
Proof.
  admit.
Admitted.

Program Definition s4: Sinv (send_nack_resp_trans proposers acceptors).
Proof.
  admit.
Admitted.


(*************************************************************)
(*                  Receive-transitions                      *)
(*************************************************************)

Program Definition r1: Rinv (receive_prepare_req_trans proposers acceptors).
Proof.
  admit.
Admitted.

Program Definition r2: Rinv (receive_accept_req_trans proposers acceptors).
Proof.
  admit.
Admitted.

Program Definition r3: Rinv (receive_promise_resp_trans proposers acceptors).
Proof.
  admit.
Admitted.

Program Definition r4: Rinv (receive_nack_resp_trans proposers acceptors).
Proof.
  admit.
Admitted.


Definition sts' := [:: SI s1; SI s2; SI s3; SI s4].
Definition rts' := [:: RI r1; RI r2; RI r3; RI r4].

Program Definition ii := @ProtocolWithInvariant.II _ _ sts' rts' _ _.

Definition paxos_with_inv := ProtocolWithIndInv ii.

End PaxosInductiveInv.
\end{lstlisting}

\vfill

\section{paxos.py}
\begin{lstlisting}[style=SourceCodeListing, language=Python]
#!/usr/bin/env python3

import subprocess
import re
import time


proposer1 = "(./PaxosMain.d.byte -me 1 -mode proposer 1 127.0.0.1 8000 2 127.0.0.1 8001 3 127.0.0.1 8002 4 127.0.0.1 8003 5 127.0.0.1 8004 &) > proposer1.log 2>&1"
proposer2 = "(./PaxosMain.d.byte -me 2 -mode proposer 1 127.0.0.1 8000 2 127.0.0.1 8001 3 127.0.0.1 8002 4 127.0.0.1 8003 5 127.0.0.1 8004 &) > proposer2.log 2>&1"
acceptor1 = "(./PaxosMain.d.byte -me 3 -mode acceptor 1 127.0.0.1 8000 2 127.0.0.1 8001 3 127.0.0.1 8002 4 127.0.0.1 8003 5 127.0.0.1 8004 &) > acceptor1.log 2>&1"
acceptor2 = "(./PaxosMain.d.byte -me 4 -mode acceptor 1 127.0.0.1 8000 2 127.0.0.1 8001 3 127.0.0.1 8002 4 127.0.0.1 8003 5 127.0.0.1 8004 &) > acceptor2.log 2>&1"
acceptor3 = "(./PaxosMain.d.byte -me 5 -mode acceptor 1 127.0.0.1 8000 2 127.0.0.1 8001 3 127.0.0.1 8002 4 127.0.0.1 8003 5 127.0.0.1 8004 &) > acceptor3.log 2>&1"

a1 = subprocess.Popen(acceptor1, shell=True)
a2 = subprocess.Popen(acceptor2, shell=True)
a3 = subprocess.Popen(acceptor3, shell=True)
p1 = subprocess.Popen(proposer1, shell=True)
p2 = subprocess.Popen(proposer2, shell=True)

consensus_achieved = True
consensus_value = None

exit_codes = [p.wait() for p in [a1, a2, a3]]
print(exit_codes)

# wait for processes to write to files
time.sleep(2)

for i in range(1, 4):
    line = subprocess.check_output(['tail', '-1', "acceptor{0}.log".format(i)])
    line = line.decode("utf-8")
    print(line.rstrip())

    pattern = r"got msg in protocol (\d) with tag = (\d), contents = \[(\d); (\d)\]"
    match = re.match(pattern, line)

    if not match or (consensus_value and consensus_value != match.group(4)):
        print("Error in acceptor %d" % i)
        consensus_achieved = False
        break
    else:
        print("Acceptor %d accepted %s" % (i, match.group(4)))
        consensus_value = match.group(4)

print("=" * 40)

if consensus_achieved:
    print("\nConsesus achieved on value: %s" % consensus_value)
else:
    print("\nConsesus not achieved")
\end{lstlisting}

\end{multicols*}

\fi

\end{landscape}
\restoregeometry
