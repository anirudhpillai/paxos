\chapter{Client Application}
As outlined in section 3.3, our approach to using Disel involved first encoding
the client application and the protocol, and then going on to write the proofs
to show that the client follows the protocol.
Implementing the client was important for a couple of reasons.
Firstly, it ensures that our adapted protocol is adequate for
implementing client applications. Having the client application working before
writing the proofs showed
that the design of the adapted protocol worked and could be used
to achieve consensus. Secondly, the client application also helped us test
our encoding of the adapted protocol. As we did not encode the inductive invariant,
the working of the client application helped us test the properties of the
adapted protocol, i.e. observing nodes achieve consensus using the protocol.

In this chapter we look at the implementation of a simple client application
that uses the proof of the adapted protocol to create an application where
a set of nodes achieve consensus on a value. The chapter first looks at how the client
application was designed and encoded in Disel. After which we look into how
Disel's \texttt{shims} runtime was used to extract the OCaml code for the
runnable client application. Then we outline how we extended the client
application by writing a wrapper around it. Finally, we look at how the client
application was proved in Disel.

\section{Modelling}
% Give a textual description with a lot of intuition about what kind of client
% we are interested in and what properties it can observe that are guaranteed by the consensus.

The main property that we want to observe from the client is the acceptors achieving
consensus on a proposal. This meant we needed to be able to see that a majority of
acceptors accepting a protocol and that all of the acceptors in the majority
accept the same proposal.

Furthermore, we needed the client application to follow the same state transition
system we designed for our adapted protocol. Thus, the correct functioning of the
client will give us confidence that our adapted protocol can be proved. This also
enables us to catch flaws in our design of the adapted protocol early on and helps
us detect things like unnecessary states early on in the process, which makes
proving the protocol easier in the later stages.

So our client was simple in that we wanted to initialise two proposers and three
acceptors and then see the acceptors achieve consensus. Each proposal that is initialised
will only try once to achieve consensus using the proposal number that it is initialised
with. If it receives a nack in the process, then it stops and does not retry to
achieve consensus with a higher proposal number. As explained in the previous
chapter, we choose this `one shot' process for the proposer in order to focus
just on the part of the protocol where consensus is achieved, i.e. where the proposer
accumulates enough promises and then sends out an accept request which then may
be accepted by each of the acceptors.

\section{Encoding client in Disel}
Having decided on the design of the client, the next step was to produce a
runnable implementation of a
proposer and an acceptor which each of the nodes in the protocol can run as programs. Here we
will only look at the implementation of the proposer, the implementation of
the acceptor follows from that and can be found in the \texttt{PaxosAcceptor.v} file.

As a proposer starts off in the \texttt{PInit} state, the runnable implementation
of a proposer needs to take in a proposal as a parameter which it will use to
initialise the proposer with. Below is the main function for the proposer.
It first sends out the prepare requests and then starts receiving responses
from the acceptors. \texttt{check\_promises} function checks that none of the
responses contain a nack request. If no nacks were received then the proposer
sends accept requests to all the acceptors by choosing the value from the
highest numbered proposal.

\begin{lstlisting}
Program Definition proposer_round (psal: proposal):
  {(e : nat)}, DHT [p, W]
  (fun i => loc i = st :-> (e, PInit psal),
   fun res m => loc m = st :-> (e, PAbort))
  :=
  Do (e <-- read_round;
      send_prepare_req_loop e psal;;
      recv_promises <-- receive_prepare_resp_loop e;
      check <-- check_promises recv_promises;
      if check
      then send_accept_reqs e (choose_highest_numbered_proposal psal recv_promises)
      else send_accept_reqs e [:: 0; 0]).
     (* If check fails then send an acc_req for (0, 0) which will never be
        accepted by any acceptor *)
\end{lstlisting}

Although, if a nack was received, the proposal still sends
accept requests with the proposal $\langle 0, 0 \rangle$. This proposal will
never be accepted by any acceptor as its proposal number is not greater than $0$.
We still need to send these accept requests as both branches of a \texttt{if}
statement need to have the same type. The distributed Hoare types and the pre and post conditions
defined are also show in the code listing. The proposer starts off in the \texttt{PInit}
state and on finishing the round, ends up in the \texttt{PAbort} state.

\section{Verfying the Client}
We will now briefly look at how the proofs were encoded.
Below we have part of the code listing for \texttt{receive\_prepare\_req\_loop}.
In the implementation you can see withing the \texttt{@while} loop that it
tries to catch an incoming prepare request and return the message if it
arrives correctly.

The pre and post conditions state that before executing this loop, the acceptor
starts in the \texttt{AInit} state and after the execution, it finishes in
the \texttt{APromised} state while holding some proposal \texttt{psal}.

\begin{lstlisting}
Definition r_prepare_req_cond (res : option proposal) := res == None.

(* Invariant relates the argument and the shape of the state *)
Definition r_prepare_req_inv (e : nat) (pinit: proposal): cont (option proposal) :=
  fun res i =>
    if res is Some psal
    then loc i = st :-> (e, APromised psal)
    else loc i = st :-> (e, AInit).

(* Loops until it receives a prepare req *)
Program Definition receive_prepare_req_loop (e : nat):
  DHT [a, W]
  (fun i => loc i = st :-> (e, AInit),
   fun res m => exists psal, (res = Some psal) /\
       (loc m = st :-> (e, APromised psal)))
  :=
  Do _ (@while a W _ _ r_prepare_req_cond (r_prepare_req_inv e) _
        (fun _ => Do _ (
           r <-- tryrecv_prepare_req;
           match r with
           | Some (from, tg, body) => ret _ _ (Some body)
           | None => ret _ _ None
           end
        )) None).
...
Next Obligation.
  move => i1/= E1.
  apply: (gh_ex (g:=([::0; 0]))).
  apply: call_rule => //r i2 [H1]H2 C2.
  rewrite /r_prepare_req_cond/r_prepare_req_inv in H1 H2.
    by case: r H1 H2 => //p _; exists p.
Qed.
\end{lstlisting}

The simple obligation we have shown in the code listing is to show that
given the pre condition holds, show that the execution of the loop will take
us to the post condition.
In order to prove it, we take an example of a prepare request with the proposal
$\langle 0, 0 \rangle$, i.e. we supply a value for the \texttt{psal} in
the post condition, and execute the statements in the loop.
We then use the definition of our loop invariant
\texttt{r\_prepare\_req\_inv} which states that as long as the incoming prepare
request is a valid proposal, you will accept it and move to the \texttt{APromised}
state. Thus, trivially completing our proof.

\section{Extraction}
Disel programs can be extracted into their corresponding OCaml definitions.
The extracted code contains modules that define the various node states
and transitions. This extracted code can then be used by a shim to create a
client application.

In order for the extraction to work, a runnable program needs to be supplied for each of
the nodes participating in the protocol. Additionally, each node needs to
be given an initial state that satisfies all the imposed invariants.

The \texttt{SimplePaxosApp.v} file uses the runnable implementations of
the proposer and acceptor and defines the code to instantiate the nodes.
The client implementation instantiates two proposers and three acceptors.
Each proposer is instantiated with a unique proposal which is required in the
\texttt{PInit} state (the state in which each proposer starts off in).

Extracting this code using Disel produces the OCaml code
that can be used to run each of the nodes. After extracting the code,
a shim file (\texttt{PaxosMain.ml}) is written that parses the arguments
supplied to it and instantiates a program specified
for the given node. Compiling the shim file produces the executable (\texttt{PaxosMain.d.byte})
which can be supplied with the right arguments to set up execution of one
of the nodes participating in the protocol.

The executable was used in a shell script (\texttt{paxos.sh})
to instantiate all the different nodes as different processes.
Below are the logs produced by one of the
proposers (node 2) on running this script. Nodes 3, 4 and 5 are acceptors while
node 1 is the other proposer proposing value 1.

\begin{lstlisting}
initial state is: [0 |-> {dstate = [1 |-> [1 |-> <heap>], 3 |-> [1 |-> <heap>],
    4 |-> [1 |-> <heap>], 5 |-> [1 |-> <heap>]]; dsoup = <>}]
World is [0 |-> <protocol with label 0>]
sending msg in protocol 0 with tag = 0, contents = [2; 2] to 3
World is [0 |-> <protocol with label 0>]
sending msg in protocol 0 with tag = 0, contents = [2; 2] to 4
World is [0 |-> <protocol with label 0>]
sending msg in protocol 0 with tag = 0, contents = [2; 2] to 5
new connection!
done processing new connection from node 3
got msg in protocol 0 with tag = 1, contents = [0; 0] from 3
new connection!
done processing new connection from node 4
got msg in protocol 0 with tag = 1, contents = [0; 0] from 4
new connection!
done processing new connection from node 5
got msg in protocol 0 with tag = 1, contents = [0; 0] from 5
World is [0 |-> <protocol with label 0>]
sending msg in protocol 0 with tag = 3, contents = [2; 2] to 3
World is [0 |-> <protocol with label 0>]
sending msg in protocol 0 with tag = 3, contents = [2; 2] to 4
World is [0 |-> <protocol with label 0>]
sending msg in protocol 0 with tag = 3, contents = [2; 2] to 5
\end{lstlisting}

The tags identify the messages types. The mapping for the tags is defined in
our protocol file (\texttt{PaxosProtocol.v}) as follows.

\begin{lstlisting}
Definition prepare_req : nat := 0.
Definition promise_resp : nat := 1.
Definition nack_resp : nat := 2.
Definition accept_req : nat := 3.
\end{lstlisting}

Thus, the logs show that the proposer first sent out prepare request
(\texttt{tag = 1}) with the proposal $\langle 2, 2 \rangle$
to the three acceptors and received promise responses back from them (\texttt{tag = 1}).
After this proposer sent out an accept request (\texttt{tag = 3}) to each of
the acceptors.

\section{Extending the extracted code}
The previous steps helped setup a simple client application but the logs
only show what happened at each individual node. Moreover, the important part is
to find out when the entire system has achieved consensus, i.e. a majority of
acceptors have accepted the same proposal.

One way of making the client do this would be to go back to the Disel code and
write the implementation of a learner and define an \textsf{accepted request} message that will
be sent from an acceptor to a learner whenever it accepts a proposal. This learner
can then tell us when the entire system has achieved consensus. This method is
time consuming because one will have to do all the encoding similar to that for
the proposer or acceptor. Although, the final output of this will be a fully
verified client where the actions of the learner also adhere to our protocol.

A quicker way to implement the same functionality is to write a
wrapper around the extracted code that enables communication with the
acceptor. The wrapper can keep checking when each of the acceptors has accepted
a proposal and can compare the values in those proposals. Using this the wrapper
can `announce' when the system has achieved consensus. An example of a wrapper, which
uses the extracted code, is implemented in the \texttt{paxos.py} file. The code
listing below shows the output of running this wrapper. The output shows
the values accepted by each acceptor and using this information, the wrapper
announce that consensus has been achieved on value $2$ as a majority of
acceptors have accepted it.

\begin{lstlisting}
got msg in protocol 0 with tag = 3, contents = [2; 2] from 2
Acceptor 1 accepted 2
got msg in protocol 0 with tag = 3, contents = [2; 2] from 2
Acceptor 2 accepted 2
got msg in protocol 0 with tag = 3, contents = [2; 2] from 2
Acceptor 3 accepted 2
========================================

Consesus achieved on value: 2
\end{lstlisting}


Writing such wrappers helps to extend the functionality of the extracted
client application and also shows that the client application can be used in
other applications where it provides a verified implementation of Paxos.


% \section{Verifying client in Disel}
% The code listing shows the spec for how an acceptor goes in a loop
% to receive a prepare request.
% \begin{lstlisting}
% (* Loops until it receives a prepare req *)
% Program Definition receive_prepare_req_loop (e : nat):
%   DHT [a, W]
%   (fun i => loc i = st :-> (e, AInit),
%    fun res m => exists psal, (res = Some psal) /\
%        (loc m = st :-> (e, APromised psal)))
%   :=
%   Do _ (@while a W _ _ r_prepare_req_cond (r_prepare_req_inv e) _
%         (fun _ => Do _ (
%            r <-- tryrecv_prepare_req;
%            match r with
%            | Some (from, tg, body) => ret _ _ (Some body)
%            | None => ret _ _ None
%            end
%         )) None).
% \end{lstlisting}
