\chapter{Requirements and Analysis}

\section{Detailed Problem Statement}

\section{Requirements}
\begin{enumerate}
  \item Adapt Paxos for encoding in Disel and devise the state-transition system for this protocol.
  \item Develop an inductive invariant for the adapted protocol that
    ensures the protocol functions correctly by imposing requirements on the global state of the system
  \item Implement a simulation of the adapted protocol with the developed state transition system.
  \item Mechanise the proof of the adapted protocol in Disel/Coq.
    Thereby, providing a library of reusable verified distributed components.
  \item Mechanise a client application of the protocol verified out of the abstract interface.
\end{enumerate}

\section{Analysis}
\subsection{Requirement 1: Adapted Protocol and State Transition System}
Having studied the Paxos protocol in detail, we adapted the protocol for
implementing it in Disel. We decided to focus on single decree paxos and to do
away with the learner for the first version of the proof in order to prove the
part of the protocol where consensus is achieved.

We developed the state transition system for the nodes in the protocol. In Paxos
each node can have different roles but we had to split up each role into different
states depending on the current data held by the node and the current function
ofthe node in the protocol. We decided on the states each node could be in and
how and when it transitions between them. This helped us come up with precondition
and postcondition for the state of each node when it transitions on receiving or
sending a message. We tried to minimise the number of transitions and the data
held in each node’s state in order to simplify the proof in Disel.

\subsection{Requirement 2: Inductive Invariant}
We also had to come up with an inductive invariant for the protocol such that if
the inductive invariant holds in some state then in holds in every state reachable
from that state. The inductive invariant was critical as it helped ensure that
the protocol functions correctly by imposing requirements on the global state of
the system. For proving the correctness of paxos we found that our invariant had
to capture when consensus is achieved on a value and also that once consensus is
achieved on a particular value, further rounds of the protocol don’t change this
value. We then also came up with a proof for how this inductive invariant holds
in our adapted protocol.

\subsection{Requirement 3: Simulation}
I have also implemented a simulator for Paxos in Python which is modeled according
to how Disel works. The simulator is based on a state-transition system like Disel
and uses separate processes to simulate different nodes in the distributed system.
In the simulator, I implemented our adapted Paxos algorithm, with the state transitions
we had decided to use with Disel. The working of the simulator gave us confidence
that our state transition system for Paxos will work correctly in Disel.

\subsection{Requirement 4: Proof}

\subsection{Requirement 5: Client Application}
After studying the Disel paper and looking at similar examples, I implemented the
core of the adapted protocol in Disel. I also implemented a client application in
Disel. The pre and post conditions from the state transition system helped me to
implement the client application in such a way to adhere with the main protocol.
Using the extraction feature in Disel and the shims runtime, I successfully
extracted a working program of the client application in OCaml.
