\chapter{Requirements and Analysis}

\section{Detailed Problem Statement}
%% TODO

\section{Requirements}
\begin{enumerate}
  \item Adapt Paxos for encoding in Disel and devise the state-transition system for this protocol.
  \item Develop an inductive invariant for the adapted protocol that
    ensures the protocol functions correctly by imposing requirements on the global state of the system
  \item Implement a simulation of the adapted protocol with the developed state transition system.
  \item Mechanise the proof of the adapted protocol in Disel/Coq.
    Thereby, providing a library of reusable verified distributed components.
  \item Mechanise a client application of the protocol verified out of the abstract interface.
\end{enumerate}

\section{Analysis}
\subsection{Requirement 1: Adapted Protocol and State Transition System}
In order to mechanise the proof of Paxos in Disel, we had to first adapt the
protocol in order to simplify the proof. There are many variants of Paxos and
we had to study the we decided to focus on single decree paxos, the variant that
was first proposed by Leslie Lamport %% TODO: Reference

We studied the protocol in detail and also decided to focus on proving the
part of the protocol that deals with achieving consensus, which is also the
main function of the protocol. For this reason we did not include the learner
in our client application or adapted protocol, nor did we focus on the part
where the chosen value is learnt by all the nodes. We instead let our inductive
invariant handle the case to detect when consensus had been achieved as the
inductive invariant can impose requriements on the global system state.

Additionally, we also had to create a state transition system for the nodes in
the protocol as Disel relies on this to impose pre and post conditions on the
states of a node. In Paxos
each node can have different roles but we had to split up each role into different
states depending on the current data held by the node and the current function
ofthe node in the protocol. We decided on the states each node could be in and
how and when it transitions between them. This helped us come up with precondition
and postcondition for the state of each node when it transitions on receiving or
sending a message. We tried to minimise the number of transitions and the data
held in each node’s state in order to simplify the proof in Disel.

We look at these in more detail in the next chapter when we talk about modelling
the protocol.

\subsection{Requirement 2: Inductive Invariant}
We also had to come up with an inductive invariant for the protocol such that if
the inductive invariant holds in some state then in holds in every state reachable
from that state. The inductive invariant was critical as it helped ensure that
the protocol functions correctly by imposing requirements on the global state of
the system. For proving the correctness of paxos we found that our invariant had
to capture when consensus is achieved on a value and also that once consensus is
achieved on a particular value, further rounds of the protocol don’t change this
value. We then also came up with a proof for how this inductive invariant holds
in our adapted protocol.

\subsection{Requirement 3: Simulation}
I also needed to implement a simulation of our adapted protocol. The simulator
must be based on a state-transition system like Disel and should be able
to simulate different nodes in the distributed system. The main reason for
implementing this simulation is to be sure that our adapted protocol, designed
in Requirement 1, will actually be provable in Disel. The simulation will enable
us to detect errors and fix them much faster than having to fixing them in the
middle of the Disel proof which is a much more time consuming to implement.
The simulator will be implemented with the same state transitions we decided
upon in Requirement 1, thus, the corrent working of the simulator will give us
confidence that our state transition system for Paxos will work correctly in Disel.

I decided to use Python to implement the simulation as that was the language I
was strongest in. I studied how the simulation for Multi Paxos was implmented
in the Paxos made moderately complex paper, %% TODO: Reference
which helped me learn how to create separate process for each node and also how
to communicate by exchanging messages between the processes.

\subsection{Requirement 4: Proof}
Having designed the state transition diagram and the inductive invariant,
the task of mechanising the proof in Disel becomes much easier.
For implementing the proofs, I needed to learn more about Coq and SSReflect.
I had to study examples of protocols proved in Disel, like the proof of the
Two Phase Commit protocol.

\subsection{Requirement 5: Client Application} %% TODO
After studying the Disel paper and looking at similar examples, I implemented the
core of the adapted protocol in Disel. I also implemented a client application in
Disel. The pre and post conditions from the state transition system helped me to
implement the client application in such a way to adhere with the main protocol.
Using the extraction feature in Disel and the shims runtime, I successfully
extracted a working program of the client application in OCaml.
