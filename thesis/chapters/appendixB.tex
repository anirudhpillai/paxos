\chapter{Interim Report}

\section{Current Status}
Having studied the Paxos protocol in detail, we adapted the protocol for implementing it in Disel. We decided to focus on single decree paxos and to do away with the learner for the first version of the proof in order to prove the part of the protocol where consensus is achieved.

We developed the state transition system for the nodes in the protocol. In Paxos each node can have different roles but we had to split up each role into different states depending on the current data held by the node and the current function of the node in the protocol. We decided on the states each node could be in and how and when it transitions between them. This helped us come up with precondition and postcondition for the state of each node when it transitions on receiving or sending a message. We tried to minimise the number of transitions and the data held in each node’s state in order to simplify the proof in Disel.

We also had to come up with an inductive invariant for the protocol such that if the inductive invariant holds in some state then in holds in every state reachable from that state. The inductive invariant was critical as it helped ensure that the protocol functions correctly by imposing requirements on the global state of the system. For proving the correctness of paxos we found that our invariant had to capture when consensus is achieved on a value and also that once consensus is achieved on a particular value, further rounds of the protocol don’t change this value. We then also came up with a proof for how this inductive invariant holds in our adapted protocol.

I have also implemented a simulator for Paxos in Python which is modeled according to how Disel works. The simulator is based on a state-transition system like Disel and uses separate processes to simulate different nodes in the distributed system. In the simulator, I implemented our adapted Paxos algorithm, with the state transitions we had decided to use with Disel. The working of the simulator gave us confidence that our state transition system for Paxos will work correctly in Disel.

After studying the Disel paper and looking at similar examples, I implemented the core of the adapted protocol in Disel. I also implemented a client application in Disel. The pre and post conditions from the state transition system helped me to implement the client application in such a way to adhere with the main protocol. Using the extraction feature in Disel and the shims runtime, I successfully extracted a working program of the client application in OCaml.

\section{Remaining Work}
Although, I have implemented the protocol in Disel, I still need to prove that the ‘coherence’ (the constraints imposed on the local state of each node and the messages exchanged) holds in the protocol. I will need to learn Coq and SSReflect in more detail to be able to finish the proofs and to understand the proofs of other protocols in Disel.

For the client implementation, I still need to prove how the implemented node roles satisfy the pre and post conditions, imposed on the state of the node, by the protocol.

I also need to mechanise the proof of our inductive invariant in Disel and prove how the pre and post conditions of each node hold with respect to the inductive invariant when the node undergoes a state transition on receiving or sending a message.

I aim to finish mechanising the proofs by the end of term (23 March) and also to get most of the project report finished by them, especially the initial sections.
