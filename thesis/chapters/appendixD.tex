\newgeometry{top=2cm, left=1cm, right=1cm, bottom=2cm}
\begin{landscape}

\chapter{Simulator Code Listing}

The complete source code for the simulator can be found at
\href{https://github.com/anirudhpillai/paxos}{https://github.com/anirudhpillai/paxos}.

\section{paxos.py}
\begin{multicols*}{2}
\begin{lstlisting}[style=SourceCodeListing,language=Python]
#!/usr/bin/env python3

import os
import signal
import sys
import time

from paxos.proposer import Proposer
from paxos.acceptor import Acceptor

NACCEPTORS = 3
NPROPOSERS = 2


class Env:
    """
    Sets up the environment and runs the protocol.
    Initialises each of the nodes in the protocol.

    Attributes:
        :procs holds all the executing processes
    """

    def __init__(self):
        self.procs = {}

    def send_msg(self, dst, msg):
        if dst in self.procs:
            self.procs[dst].deliver(msg)

    def add_proc(self, proc):
        self.procs[proc.id] = proc
        proc.start()

    def remove_proc(self, pid):
        del self.procs[pid]

    def run(self):
        proposers = range(1, NPROPOSERS + 1)
        acceptors = range(NPROPOSERS, NPROPOSERS + NACCEPTORS)

        for i in acceptors:
            pid = "Acceptor %d" % i
            Acceptor(self, i)

        for i in proposers:
            pid = "Proposer %d" % i
            Proposer(self, acceptors, i, i)

    def terminate_handler(self, signal, frame):
        self._graceful_exit()

    def _graceful_exit(self, exit_code=0):
        sys.stdout.flush()
        sys.stderr.flush()
        os._exit(exit_code)


def main():
    e = Env()
    e.run()
    signal.signal(signal.SIGINT, e.terminate_handler)
    signal.signal(signal.SIGTERM, e.terminate_handler)
    signal.pause()


main()
\end{lstlisting}
\end{multicols*}

\newpage

\begin{multicols*}{2}
\section{paxos/process.py}
\begin{lstlisting}[style=SourceCodeListing,language=Python]
import multiprocessing
import threading


class Process(threading.Thread):
    """
    Attributes:
        :state  current state of the Acceptor
        :id     id of the Acceptor
        :env    environment this Acceptor is running in
    """

    def __init__(self, env, id):
        super(Process, self).__init__()
        self.inbox = multiprocessing.Manager().Queue()
        self.env = env
        self.id = id

    def run(self):
        try:
            self.body()
            self.env.remove_proc(self.id)
        except EOFError:
            print("Exiting...")

    def get_next_msg(self):
        return self.inbox.get()

    def send_msg(self, dst, msg):
        self.env.send_msg(dst, msg)

    def deliver(self, msg):
        self.inbox.put(msg)
\end{lstlisting}

\vfill\null
\columnbreak

\section{paxos/message.py}
\begin{lstlisting}[style=SourceCodeListing,language=Python]
"""
This file defines the Message super class and all the other
types of messages used in the adapted Simple Paxos protocol.
"""


class Message:
    """
    Message super class which all the differnt message types inherit.

    Attributes:
        :src       sender of the message
        :proposal  proposal contained in the message
    """

    def __init__(self, src, proposal):
        self.src = src
        self.proposal = proposal

    def __str__(self):
        return str(self.__dict__)


class PrepareRequestMessage(Message):
    def __init__(self, src, proposal):
        Message.__init__(self, src, proposal)


class AcceptRequestMessage(Message):
    def __init__(self, src, proposal):
        Message.__init__(self, src, proposal)


class PromiseResponseMessage(Message):
    def __init__(self, src, proposal):
        Message.__init__(self, src, proposal)


class NackResponseMessage(Message):
    def __init__(self, src):
        Message.__init__(self, src, (-1, -1))
\end{lstlisting}
\end{multicols*}

\newpage


\section{paxos/proposer.py}
\begin{multicols*}{2}
\begin{lstlisting}[style=SourceCodeListing,language=Python]
from .process import Process
from .message import AcceptRequestMessage, PrepareRequestMessage, \
    PromiseResponseMessage, NackResponseMessage


class Proposer(Process):
    """
    Implementation of the Proposer in the adapted Simple Paxos protocol.

    Attributes:
        :state      current state of the Proposer
        :acceptors  set of acceptors in the current environment
        :env        environment this Proposer is running in
    """

    def __init__(self, env, acceptors, p_no, p_val):
        Process.__init__(self, env, p_no)
        self.state = ("PInit", p_no, p_val)
        self.acceptors = acceptors
        self.env = env
        self.env.add_proc(self)

    def send_prepare_req(self):
        _, p_no, p_val = self.state
        for acceptor in self.acceptors:
            self.send_msg(
                acceptor,
                PrepareRequestMessage(p_no, (p_no, p_val))
            )

    def send_accept_req(self):
        _, p_no, p_val = self.state
        for acceptor in self.acceptors:
            self.send_msg(
                acceptor,
                AcceptRequestMessage(p_no, (p_no, p_val))
            )

    def body(self):
        _, p_no, p_val = self.state

        self.send_prepare_req()
        self.state = ("PWaitPrepResp", [], p_no, p_val)

        while True:
            msg = self.get_next_msg()
            if isinstance(msg, PromiseResponseMessage):
                if self.state[0] == "PWaitPrepResp":
                    src = msg.src
                    recv_p_no, recv_p_val = msg.proposal
                    recv_promises = self.state[1]

                    if src not in map(lambda x: x[0], recv_promises):
                        recv_promises.append((src, recv_p_no, recv_p_val))
                        if sorted(
                            map(lambda x: x[0], recv_promises)
                        ) == sorted(self.acceptors):
                            recv_promises.sort(key=lambda x: x[1])
                            highest_numbered_value = recv_promises[0][2]
                            self.state = (
                                "PSentAccReq", p_no, highest_numbered_value
                            )
                            self.send_accept_req()
                            print("Exiting proposer", p_no)
                            break
                        else:
                            self.state = (
                                "PWaitPrepResp", recv_promises, p_no, p_val
                            )
            elif isinstance(msg, NackResponseMessage):
                print("Exiting proposer", p_no)
                break
\end{lstlisting}
\end{multicols*}

\newpage

\section{paxos/acceptor.py}
\begin{multicols*}{2}
\begin{lstlisting}[style=SourceCodeListing,language=Python]
from .process import Process
from .message import AcceptRequestMessage, PrepareRequestMessage, \
    PromiseResponseMessage, NackResponseMessage


class Acceptor(Process):
    """
    Implementation of the Acceptor in the adapted Simple Paxos protocol.

    Attributes:
        :state  current state of the Acceptor
        :id     id of the Acceptor
        :env    environment this Acceptor is running in
    """

    def __init__(self, env, id):
        Process.__init__(self, env, id)
        self.state = ("AInit",)
        self.env = env
        self.env.add_proc(self)
        self.id = id

    def send_promise_resp(self, to):
        p_no, p_val = self.state[1]
        self.send_msg(to, PromiseResponseMessage(self.id, (p_no, p_val)))

    def send_nack_resp(self, to):
        self.send_msg(to, NackResponseMessage(self.id))

    def body(self):
        while True:
            msg = self.get_next_msg()
            p_no, p_val = msg.proposal

            if isinstance(msg, PrepareRequestMessage):
                if self.state[0] == "AInit":
                    self.state = ("APromised", msg.proposal)
                    self.send_promise_resp(p_no)
                elif self.state[0] == "APromised":
                    promised_no, _ = self.state[1]
                    if p_val < promised_no:
                        self.send_nack_resp(p_no)
                    else:
                        self.state = ("APromised", msg.proposal)
                        self.send_promise_resp(p_no)
                else:
                    self.send_nack_resp(p_no)
            elif isinstance(msg, AcceptRequestMessage):
                if self.state[0] == "AInit":
                    self.state = ("AAccepted", msg.proposal)
                else:
                    promised_no, _ = self.state[1]
                    if p_val >= promised_no:
                        self.state = ("AAccepted", msg.proposal)
                        print("Node %d has state %s" % (self.id, self.state))
\end{lstlisting}

\end{multicols*}
\end{landscape}
\restoregeometry
