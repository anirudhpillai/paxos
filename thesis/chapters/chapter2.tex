\chapter{Background}
This chapter lays down all the previous research which the project builds on. Before
going over the design decisions on the project we first need to understand this
background information and look at related work to see different approaches used
to solve the problem.

\newpage

\section{Distributed Systems}
%% TODO: Find nice intro

A distributed system is a model in which processes running on running different
computers, which are connected together in a network, exchange messages to coordinate
their action, often resulting in the user thinking of the entire system as one single
unified computer.

A computer in the distibuted system is also alternatively referred to as a
processor or a node in the system. Each node in a distributed systems has its
own memory.

%% TODO: Structured in terms of client and servers

We will now go over a few concepts of distributed systems which will help us understand
the characteristics of the protocols that run on these systems. This will lay down the
groundwork for us to understand the Paxos protocol on which this project is based.

\subsubsection{Asynchronous Environment}
An asynchronous distributed system is one where there are no guarantees about the
timing and order in which events occur.

The clocks of each of the process in the system can be out of sync and may not be
accurate. Therefore, there can be no guarantees about the order in which events occur.

Further, messages sent by one process to another can be delayed for an arbitary period of time.

A protocol running in an asynchronous enviroment has to account for these conditions
in its design and try to achieve its goal without the guarantees of timed events.
An asynchronous environment is very common for a real world distributed system but
it also makes reasoning about the system harder because of the aforementioned properties.

\subsubsection{Faults Tolerance}
A fault tolerant distributed system is one which can continue to function correctly
despite the failure of some of its components. A 'failure' of a node or 'fault' in a node
means any unexpected behaviour from that node eg. not responding to messages, sending
corrupted messages.

Fault tolerance is one of the main reasons for using a distributed system as it
increases the chances of your application continuing to functioning correctly and
makes it more dependable. As Netflix mention on their blog
'Fault Tolerance is a Requirement, Not a Feature'.
%% TODO: Reference https://medium.com/netflix-techblog/fault-tolerance-in-a-high-volume-distributed-system-91ab4faae74a
With their Netflix API receiving more than 1 billion requests a day, they expect
that it is guaranteed that some of the components of their distributed system will fail.
Using a fault tolerant distributed system they are able to ensure that a small failure
in some components doesn't hinder the performance of the overall system, hence,
enabling them to achieve their uptime metrics.

Fault tolerant distributed system protocols are protocols which achieve their
goals despite the failure of some of the components of the distributed system they run on.
The protocol accounts for the failures and generally specifies the maximum number of
failures and the types of failures it can handle before it stops functioning correctly.

\subsubsection{State Machine Replication}
%% TODO: Reference https://www.cs.cornell.edu/fbs/publications/SMSurvey.pdf

In a client server model, the easiest way to implement it is to use one single server
which handles all the client request. Obviously this isn't the most robust solution
as if the single server fails, so does your service. To overcome the problem you
use a collection of servers each of which is a replica of the original single server and
ensure that each of these 'replicas' fails independantly, without effecting the other replicas.
This adds more fault tolerance.

State Machine Replication is method for creating a fault tolerant distributed system
by replicating servers and using protocols to coordinate the interactions of these
replicated servers with the client.

A State Machine $M$ can be defined as $M = \lbrace q_0, Q, I, O, \delta, \gamma \rbrace$ where \\
$q_0$ is the starting state \\
$Q$ is the set of all possible states.
$I$ is set of all valid inputs
$O$ is the set of all valide outputs
$\delta$ is the state transition function, $\delta : I x Q -> Q$
$\gamma$ is the output function, $\gamma : I x Q -> O$

%% TODO: Complete State Machine Replication Information

The method of modelling a distributed system protocol as state transition system
is very common and is a critical component of this project as well.

%% TODO: Cover Byzantine Faults

\subsubsection{Consensus Protocols}
For handling faults in your distributed system you need to have replication.
This leads to the problem of making all these replicas agree with each other
to keep them consistent. Consensus protocols try to solve this problem.

Consensus protocols are the family of distibuted systems protocols which aim to
make a distributed network of processes agree on one result.

These protocols are of interest because of their numerous real world applications.
Let us take the example of a distributed database, which is a critical part of almost
all large scale real world applications. This distributed database will run
over a network of computers and everytime you use the database you aren't guaranteed
to be served by the same computer.

Suppose you add a file to the database. This action is performed by the processor that
was serving you 'add' request. Later when you want to retrieve the file from the database
you might be served by a different computer that did not perform the 'add' request. Inorder
for the new computer to know that the file exists in the database, you will need to use a
consensus protocol which helps all the computers in the network (which handle user
requests) agree upon the result that the file has been added to the database.

Popular consensus protocols include PageRank used by Google and the Blockchain
consensus protocol. George and Ilya verified a subset of the protocol in Coq in
their Toychain paper.
%% TODO: references


\section{Paxos}
Having understood the the main concepts behind distributed system protocols, we can
now finally get to the protocol at the heart of this project. Paxos is a family of
asynchronous, fault tolerant, consensus protocol which achieves consensus in a network
of unrealiable processes as long as a majority of them don't fail.

Paxos is used for state machine replication and helps all the replicas achieve
consensus on a result.

\section{Disel}

\section{Related Work}
