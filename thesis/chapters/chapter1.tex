\chapter{Introduction}

\section{The Problem}
This project aims to implement a library of reusable verified distributed components,
based on the classical family of fault-tolerant asynchronous Paxos-like consensus protocol.
The project will use Disel, a framework for compositional verification of distributed
protocols built on top of the Coq proof assistant, to verify correctness of the
implemented components.

\section{Aims and Goals}
I have highlighted the aims and goals separately. The aims are what I want to
achieve out of undertaking this project and the goals are the things that this
project tries to achieve.

\subsection{Aims}
\begin{enumerate}
\itemsep0em
  \item Learn about distributed system protocols
  \item Contribute to open source software
\end{enumerate}

\vspace{-5mm}
\subsection{Goals}
\begin{enumerate}
\itemsep0em
  \item Read about and understand the classical Paxos-like consensus algorithms.
  \item Develop state transition systems for the algorithms and identify the invariants that need to be preserved during the operation of the algorithm.
  \item Implement a simulation of the protocols in Python.
  \item Formulate the implemented protocols in Disel by using the developed state-transition systems.
  \item Mechanise the proofs of the identified protocol invariants in Disel/Coq.
  \item Add additional communication channels and prove composite invariants.
  \item Provide an abstract specification of the protocol, usable by third-party clients.
  \item Mechanise a client application of the protocol verified out of the abstract interface.
\end{enumerate}

\vspace{-4mm}
\section{Project Overview}
% Give an overview of how you carried out the project (e.g., an iterative
% approach).

\section{Report Overview}
% A brief overview of the rest of the chapters in the report (a guide to
% the reader of the overall structure of the report).
